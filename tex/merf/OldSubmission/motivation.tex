\setcode{utf8}
\setarab
\transfalse
\begin{figure*}[tb!]
\renewcommand{\arraystretch}{1.5}% Wider
\resizebox{2\columnwidth}{!}{
\begin{tabular}{cc}
\resizebox{\columnwidth}{!}{
	\begin{tabular}{c|c|c|c}
		MBF & description & formula & matches\\ \hline
        N & name of person & $category~=~Name\_of\_Person$ & $n_1,n_2,n_3$ \\ \hline
        P & name of place & $category~=~Name\_of\_Place$ & $p_1,p_2,...,p_7$ \\ \hline
        R & relative position & $stem\in$ \{\RL{قرب},\RL{في},\dots\} & $r_1,r_2,r_3,r_4$ \\ \hline
        U & numerical term & $stem\in$ \{\RL{أول},\RL{ثاني},\dots\} & $u_1,u_2$ \\
    \end{tabular}
} 
&
\resizebox{\columnwidth}{!}{
\begin{tabular}{cc}
\relsize{-2} % Graphic for TeX using PGF
% Title: /home/ameen/Desktop/match1.dia
% Creator: Dia v0.97.1
% CreationDate: Sat Jan 11 00:35:23 2014
% For: ameen
% \usepackage{tikz}
% The following commands are not supported in PSTricks at present
% We define them conditionally, so when they are implemented,
% this pgf file will use them.
\ifx\du\undefined
  \newlength{\du}
\fi
\setlength{\du}{15\unitlength}
\begin{tikzpicture}
\pgftransformxscale{1.000000}
\pgftransformyscale{-1.000000}
\definecolor{dialinecolor}{rgb}{0.000000, 0.000000, 0.000000}
\pgfsetstrokecolor{dialinecolor}
\definecolor{dialinecolor}{rgb}{1.000000, 1.000000, 1.000000}
\pgfsetfillcolor{dialinecolor}
\definecolor{dialinecolor}{rgb}{1.000000, 1.000000, 1.000000}
\pgfsetfillcolor{dialinecolor}
\fill (23.657500\du,7.687500\du)--(23.657500\du,8.537500\du)--(24.632500\du,8.537500\du)--(24.632500\du,7.687500\du)--cycle;
\pgfsetlinewidth{0.020000\du}
\pgfsetdash{}{0pt}
\pgfsetdash{}{0pt}
\pgfsetmiterjoin
\definecolor{dialinecolor}{rgb}{0.000000, 0.000000, 0.000000}
\pgfsetstrokecolor{dialinecolor}
\draw (23.657500\du,7.687500\du)--(23.657500\du,8.537500\du)--(24.632500\du,8.537500\du)--(24.632500\du,7.687500\du)--cycle;
% setfont left to latex
\definecolor{dialinecolor}{rgb}{0.000000, 0.000000, 0.000000}
\pgfsetstrokecolor{dialinecolor}
\node at (24.145000\du,8.233056\du){()};
\definecolor{dialinecolor}{rgb}{1.000000, 1.000000, 1.000000}
\pgfsetfillcolor{dialinecolor}
\pgfpathellipse{\pgfpoint{25.225101\du}{9.941465\du}}{\pgfpoint{1.185301\du}{0\du}}{\pgfpoint{0\du}{0.726265\du}}
\pgfusepath{fill}
\pgfsetlinewidth{0.020000\du}
\pgfsetdash{}{0pt}
\pgfsetdash{}{0pt}
\pgfsetmiterjoin
\definecolor{dialinecolor}{rgb}{0.000000, 0.000000, 0.000000}
\pgfsetstrokecolor{dialinecolor}
\pgfpathellipse{\pgfpoint{25.225101\du}{9.941465\du}}{\pgfpoint{1.185301\du}{0\du}}{\pgfpoint{0\du}{0.726265\du}}
\pgfusepath{stroke}
% setfont left to latex
\definecolor{dialinecolor}{rgb}{0.000000, 0.000000, 0.000000}
\pgfsetstrokecolor{dialinecolor}
\node at (25.225101\du,10.062021\du){r1=\RL{بالقرب}};
\definecolor{dialinecolor}{rgb}{1.000000, 1.000000, 1.000000}
\pgfsetfillcolor{dialinecolor}
\pgfpathellipse{\pgfpoint{22.605219\du}{13.362071\du}}{\pgfpoint{1.319819\du}{0\du}}{\pgfpoint{0\du}{0.707771\du}}
\pgfusepath{fill}
\pgfsetlinewidth{0.020000\du}
\pgfsetdash{}{0pt}
\pgfsetdash{}{0pt}
\pgfsetmiterjoin
\definecolor{dialinecolor}{rgb}{0.000000, 0.000000, 0.000000}
\pgfsetstrokecolor{dialinecolor}
\pgfpathellipse{\pgfpoint{22.605219\du}{13.362071\du}}{\pgfpoint{1.319819\du}{0\du}}{\pgfpoint{0\du}{0.707771\du}}
\pgfusepath{stroke}
% setfont left to latex
\definecolor{dialinecolor}{rgb}{0.000000, 0.000000, 0.000000}
\pgfsetstrokecolor{dialinecolor}
\node at (22.605219\du,13.482627\du){p2=\RL{التقاطع}};
\definecolor{dialinecolor}{rgb}{1.000000, 1.000000, 1.000000}
\pgfsetfillcolor{dialinecolor}
\fill (22.757500\du,9.782730\du)--(22.757500\du,10.667730\du)--(23.757500\du,10.667730\du)--(23.757500\du,9.782730\du)--cycle;
\pgfsetlinewidth{0.020000\du}
\pgfsetdash{}{0pt}
\pgfsetdash{}{0pt}
\pgfsetmiterjoin
\definecolor{dialinecolor}{rgb}{0.000000, 0.000000, 0.000000}
\pgfsetstrokecolor{dialinecolor}
\draw (22.757500\du,9.782730\du)--(22.757500\du,10.667730\du)--(23.757500\du,10.667730\du)--(23.757500\du,9.782730\du)--cycle;
% setfont left to latex
\definecolor{dialinecolor}{rgb}{0.000000, 0.000000, 0.000000}
\pgfsetstrokecolor{dialinecolor}
\node at (23.257500\du,10.345786\du){\^{}2};
\definecolor{dialinecolor}{rgb}{1.000000, 1.000000, 1.000000}
\pgfsetfillcolor{dialinecolor}
\pgfpathellipse{\pgfpoint{23.312350\du}{11.641042\du}}{\pgfpoint{0.695150\du}{0\du}}{\pgfpoint{0\du}{0.488202\du}}
\pgfusepath{fill}
\pgfsetlinewidth{0.020000\du}
\pgfsetdash{}{0pt}
\pgfsetdash{}{0pt}
\pgfsetmiterjoin
\definecolor{dialinecolor}{rgb}{0.000000, 0.000000, 0.000000}
\pgfsetstrokecolor{dialinecolor}
\pgfpathellipse{\pgfpoint{23.312350\du}{11.641042\du}}{\pgfpoint{0.695150\du}{0\du}}{\pgfpoint{0\du}{0.488202\du}}
\pgfusepath{stroke}
% setfont left to latex
\definecolor{dialinecolor}{rgb}{0.000000, 0.000000, 0.000000}
\pgfsetstrokecolor{dialinecolor}
\node at (23.312350\du,11.761598\du){\RL{من}};
\definecolor{dialinecolor}{rgb}{1.000000, 1.000000, 1.000000}
\pgfsetfillcolor{dialinecolor}
\pgfpathellipse{\pgfpoint{20.061250\du}{13.438719\du}}{\pgfpoint{1.046250\du}{0\du}}{\pgfpoint{0\du}{0.631119\du}}
\pgfusepath{fill}
\pgfsetlinewidth{0.020000\du}
\pgfsetdash{}{0pt}
\pgfsetdash{}{0pt}
\pgfsetmiterjoin
\definecolor{dialinecolor}{rgb}{0.000000, 0.000000, 0.000000}
\pgfsetstrokecolor{dialinecolor}
\pgfpathellipse{\pgfpoint{20.061250\du}{13.438719\du}}{\pgfpoint{1.046250\du}{0\du}}{\pgfpoint{0\du}{0.631119\du}}
\pgfusepath{stroke}
% setfont left to latex
\definecolor{dialinecolor}{rgb}{0.000000, 0.000000, 0.000000}
\pgfsetstrokecolor{dialinecolor}
\node at (20.061250\du,13.559275\du){u1=\RL{الأول}};
\definecolor{dialinecolor}{rgb}{1.000000, 1.000000, 1.000000}
\pgfsetfillcolor{dialinecolor}
\pgfpathellipse{\pgfpoint{25.370283\du}{13.399620\du}}{\pgfpoint{1.072583\du}{0\du}}{\pgfpoint{0\du}{0.670220\du}}
\pgfusepath{fill}
\pgfsetlinewidth{0.020000\du}
\pgfsetdash{}{0pt}
\pgfsetdash{}{0pt}
\pgfsetmiterjoin
\definecolor{dialinecolor}{rgb}{0.000000, 0.000000, 0.000000}
\pgfsetstrokecolor{dialinecolor}
\pgfpathellipse{\pgfpoint{25.370283\du}{13.399620\du}}{\pgfpoint{1.072583\du}{0\du}}{\pgfpoint{0\du}{0.670220\du}}
\pgfusepath{stroke}
% setfont left to latex
\definecolor{dialinecolor}{rgb}{0.000000, 0.000000, 0.000000}
\pgfsetstrokecolor{dialinecolor}
\node at (25.370283\du,13.520176\du){n1=\RL{خليفة}};
\definecolor{dialinecolor}{rgb}{1.000000, 1.000000, 1.000000}
\pgfsetfillcolor{dialinecolor}
\pgfpathellipse{\pgfpoint{27.556244\du}{13.449916\du}}{\pgfpoint{0.861244\du}{0\du}}{\pgfpoint{0\du}{0.619916\du}}
\pgfusepath{fill}
\pgfsetlinewidth{0.020000\du}
\pgfsetdash{}{0pt}
\pgfsetdash{}{0pt}
\pgfsetmiterjoin
\definecolor{dialinecolor}{rgb}{0.000000, 0.000000, 0.000000}
\pgfsetstrokecolor{dialinecolor}
\pgfpathellipse{\pgfpoint{27.556244\du}{13.449916\du}}{\pgfpoint{0.861244\du}{0\du}}{\pgfpoint{0\du}{0.619916\du}}
\pgfusepath{stroke}
% setfont left to latex
\definecolor{dialinecolor}{rgb}{0.000000, 0.000000, 0.000000}
\pgfsetstrokecolor{dialinecolor}
\node at (27.556244\du,13.570472\du){p1=\RL{برج}};
\definecolor{dialinecolor}{rgb}{1.000000, 1.000000, 1.000000}
\pgfsetfillcolor{dialinecolor}
\fill (26.957500\du,9.717730\du)--(26.957500\du,10.667730\du)--(27.882500\du,10.667730\du)--(27.882500\du,9.717730\du)--cycle;
\pgfsetlinewidth{0.020000\du}
\pgfsetdash{}{0pt}
\pgfsetdash{}{0pt}
\pgfsetmiterjoin
\definecolor{dialinecolor}{rgb}{0.000000, 0.000000, 0.000000}
\pgfsetstrokecolor{dialinecolor}
\draw (26.957500\du,9.717730\du)--(26.957500\du,10.667730\du)--(27.882500\du,10.667730\du)--(27.882500\du,9.717730\du)--cycle;
% setfont left to latex
\definecolor{dialinecolor}{rgb}{0.000000, 0.000000, 0.000000}
\pgfsetstrokecolor{dialinecolor}
\node at (27.420000\du,10.313286\du){$+$};
\definecolor{dialinecolor}{rgb}{1.000000, 1.000000, 1.000000}
\pgfsetfillcolor{dialinecolor}
\fill (27.082500\du,11.203540\du)--(27.082500\du,12.078540\du)--(27.982500\du,12.078540\du)--(27.982500\du,11.203540\du)--cycle;
\pgfsetlinewidth{0.020000\du}
\pgfsetdash{}{0pt}
\pgfsetdash{}{0pt}
\pgfsetmiterjoin
\definecolor{dialinecolor}{rgb}{0.000000, 0.000000, 0.000000}
\pgfsetstrokecolor{dialinecolor}
\draw (27.082500\du,11.203540\du)--(27.082500\du,12.078540\du)--(27.982500\du,12.078540\du)--(27.982500\du,11.203540\du)--cycle;
% setfont left to latex
\definecolor{dialinecolor}{rgb}{0.000000, 0.000000, 0.000000}
\pgfsetstrokecolor{dialinecolor}
\node at (27.532500\du,11.761596\du){$|$};
\definecolor{dialinecolor}{rgb}{1.000000, 1.000000, 1.000000}
\pgfsetfillcolor{dialinecolor}
\fill (21.507500\du,11.178540\du)--(21.507500\du,12.103540\du)--(22.457500\du,12.103540\du)--(22.457500\du,11.178540\du)--cycle;
\pgfsetlinewidth{0.020000\du}
\pgfsetdash{}{0pt}
\pgfsetdash{}{0pt}
\pgfsetmiterjoin
\definecolor{dialinecolor}{rgb}{0.000000, 0.000000, 0.000000}
\pgfsetstrokecolor{dialinecolor}
\draw (21.507500\du,11.178540\du)--(21.507500\du,12.103540\du)--(22.457500\du,12.103540\du)--(22.457500\du,11.178540\du)--cycle;
% setfont left to latex
\definecolor{dialinecolor}{rgb}{0.000000, 0.000000, 0.000000}
\pgfsetstrokecolor{dialinecolor}
\node at (21.982500\du,11.761596\du){$|$};
\definecolor{dialinecolor}{rgb}{1.000000, 1.000000, 1.000000}
\pgfsetfillcolor{dialinecolor}
\fill (20.382500\du,9.717730\du)--(20.382500\du,10.667730\du)--(21.307500\du,10.667730\du)--(21.307500\du,9.717730\du)--cycle;
\pgfsetlinewidth{0.020000\du}
\pgfsetdash{}{0pt}
\pgfsetdash{}{0pt}
\pgfsetmiterjoin
\definecolor{dialinecolor}{rgb}{0.000000, 0.000000, 0.000000}
\pgfsetstrokecolor{dialinecolor}
\draw (20.382500\du,9.717730\du)--(20.382500\du,10.667730\du)--(21.307500\du,10.667730\du)--(21.307500\du,9.717730\du)--cycle;
% setfont left to latex
\definecolor{dialinecolor}{rgb}{0.000000, 0.000000, 0.000000}
\pgfsetstrokecolor{dialinecolor}
\node at (20.845000\du,10.313286\du){$+$};
\pgfsetlinewidth{0.020000\du}
\pgfsetdash{}{0pt}
\pgfsetdash{}{0pt}
\pgfsetbuttcap
{
\definecolor{dialinecolor}{rgb}{0.000000, 0.000000, 0.000000}
\pgfsetfillcolor{dialinecolor}
% was here!!!
\pgfsetarrowsend{latex}
\definecolor{dialinecolor}{rgb}{0.000000, 0.000000, 0.000000}
\pgfsetstrokecolor{dialinecolor}
\draw (24.642726\du,8.428648\du)--(26.953057\du,9.896135\du);
}
\pgfsetlinewidth{0.020000\du}
\pgfsetdash{}{0pt}
\pgfsetdash{}{0pt}
\pgfsetbuttcap
{
\definecolor{dialinecolor}{rgb}{0.000000, 0.000000, 0.000000}
\pgfsetfillcolor{dialinecolor}
% was here!!!
\pgfsetarrowsend{latex}
\definecolor{dialinecolor}{rgb}{0.000000, 0.000000, 0.000000}
\pgfsetstrokecolor{dialinecolor}
\draw (27.457628\du,10.677150\du)--(27.497728\du,11.193393\du);
}
\pgfsetlinewidth{0.020000\du}
\pgfsetdash{}{0pt}
\pgfsetdash{}{0pt}
\pgfsetbuttcap
{
\definecolor{dialinecolor}{rgb}{0.000000, 0.000000, 0.000000}
\pgfsetfillcolor{dialinecolor}
% was here!!!
\pgfsetarrowsend{latex}
\definecolor{dialinecolor}{rgb}{0.000000, 0.000000, 0.000000}
\pgfsetstrokecolor{dialinecolor}
\draw (27.538372\du,12.088401\du)--(27.547978\du,12.820166\du);
}
\definecolor{dialinecolor}{rgb}{1.000000, 1.000000, 1.000000}
\pgfsetfillcolor{dialinecolor}
\fill (24.832500\du,11.191040\du)--(24.832500\du,12.091040\du)--(25.807500\du,12.091040\du)--(25.807500\du,11.191040\du)--cycle;
\pgfsetlinewidth{0.020000\du}
\pgfsetdash{}{0pt}
\pgfsetdash{}{0pt}
\pgfsetmiterjoin
\definecolor{dialinecolor}{rgb}{0.000000, 0.000000, 0.000000}
\pgfsetstrokecolor{dialinecolor}
\draw (24.832500\du,11.191040\du)--(24.832500\du,12.091040\du)--(25.807500\du,12.091040\du)--(25.807500\du,11.191040\du)--cycle;
% setfont left to latex
\definecolor{dialinecolor}{rgb}{0.000000, 0.000000, 0.000000}
\pgfsetstrokecolor{dialinecolor}
\node at (25.320000\du,11.761596\du){$|$};
\pgfsetlinewidth{0.020000\du}
\pgfsetdash{}{0pt}
\pgfsetdash{}{0pt}
\pgfsetbuttcap
{
\definecolor{dialinecolor}{rgb}{0.000000, 0.000000, 0.000000}
\pgfsetfillcolor{dialinecolor}
% was here!!!
\pgfsetarrowsend{latex}
\definecolor{dialinecolor}{rgb}{0.000000, 0.000000, 0.000000}
\pgfsetstrokecolor{dialinecolor}
\draw (26.947808\du,10.518388\du)--(25.817314\du,11.298056\du);
}
\pgfsetlinewidth{0.020000\du}
\pgfsetdash{}{0pt}
\pgfsetdash{}{0pt}
\pgfsetbuttcap
{
\definecolor{dialinecolor}{rgb}{0.000000, 0.000000, 0.000000}
\pgfsetfillcolor{dialinecolor}
% was here!!!
\pgfsetarrowsend{latex}
\definecolor{dialinecolor}{rgb}{0.000000, 0.000000, 0.000000}
\pgfsetstrokecolor{dialinecolor}
\draw (25.333062\du,12.097859\du)--(25.350826\du,12.719115\du);
}
\pgfsetlinewidth{0.020000\du}
\pgfsetdash{}{0pt}
\pgfsetdash{}{0pt}
\pgfsetbuttcap
{
\definecolor{dialinecolor}{rgb}{0.000000, 0.000000, 0.000000}
\pgfsetfillcolor{dialinecolor}
% was here!!!
\pgfsetarrowsend{latex}
\definecolor{dialinecolor}{rgb}{0.000000, 0.000000, 0.000000}
\pgfsetstrokecolor{dialinecolor}
\draw (24.400258\du,8.544736\du)--(24.816635\du,9.249798\du);
}
\pgfsetlinewidth{0.020000\du}
\pgfsetdash{}{0pt}
\pgfsetdash{}{0pt}
\pgfsetbuttcap
{
\definecolor{dialinecolor}{rgb}{0.000000, 0.000000, 0.000000}
\pgfsetfillcolor{dialinecolor}
% was here!!!
\pgfsetarrowsend{latex}
\definecolor{dialinecolor}{rgb}{0.000000, 0.000000, 0.000000}
\pgfsetstrokecolor{dialinecolor}
\draw (23.962560\du,8.546806\du)--(23.447524\du,9.772871\du);
}
\pgfsetlinewidth{0.020000\du}
\pgfsetdash{}{0pt}
\pgfsetdash{}{0pt}
\pgfsetbuttcap
{
\definecolor{dialinecolor}{rgb}{0.000000, 0.000000, 0.000000}
\pgfsetfillcolor{dialinecolor}
% was here!!!
\pgfsetarrowsend{latex}
\definecolor{dialinecolor}{rgb}{0.000000, 0.000000, 0.000000}
\pgfsetstrokecolor{dialinecolor}
\draw (23.275029\du,10.677695\du)--(23.293053\du,11.142950\du);
}
\definecolor{dialinecolor}{rgb}{1.000000, 1.000000, 1.000000}
\pgfsetfillcolor{dialinecolor}
\fill (19.635000\du,11.178540\du)--(19.635000\du,12.103540\du)--(20.585000\du,12.103540\du)--(20.585000\du,11.178540\du)--cycle;
\pgfsetlinewidth{0.020000\du}
\pgfsetdash{}{0pt}
\pgfsetdash{}{0pt}
\pgfsetmiterjoin
\definecolor{dialinecolor}{rgb}{0.000000, 0.000000, 0.000000}
\pgfsetstrokecolor{dialinecolor}
\draw (19.635000\du,11.178540\du)--(19.635000\du,12.103540\du)--(20.585000\du,12.103540\du)--(20.585000\du,11.178540\du)--cycle;
% setfont left to latex
\definecolor{dialinecolor}{rgb}{0.000000, 0.000000, 0.000000}
\pgfsetstrokecolor{dialinecolor}
\node at (20.110000\du,11.761596\du){$|$};
\pgfsetlinewidth{0.020000\du}
\pgfsetdash{}{0pt}
\pgfsetdash{}{0pt}
\pgfsetbuttcap
{
\definecolor{dialinecolor}{rgb}{0.000000, 0.000000, 0.000000}
\pgfsetfillcolor{dialinecolor}
% was here!!!
\pgfsetarrowsend{latex}
\definecolor{dialinecolor}{rgb}{0.000000, 0.000000, 0.000000}
\pgfsetstrokecolor{dialinecolor}
\draw (21.225463\du,10.677150\du)--(21.611202\du,11.168288\du);
}
\pgfsetlinewidth{0.020000\du}
\pgfsetdash{}{0pt}
\pgfsetdash{}{0pt}
\pgfsetbuttcap
{
\definecolor{dialinecolor}{rgb}{0.000000, 0.000000, 0.000000}
\pgfsetfillcolor{dialinecolor}
% was here!!!
\pgfsetarrowsend{latex}
\definecolor{dialinecolor}{rgb}{0.000000, 0.000000, 0.000000}
\pgfsetstrokecolor{dialinecolor}
\draw (20.599163\du,10.677150\du)--(20.349916\du,11.168288\du);
}
\pgfsetlinewidth{0.020000\du}
\pgfsetdash{}{0pt}
\pgfsetdash{}{0pt}
\pgfsetbuttcap
{
\definecolor{dialinecolor}{rgb}{0.000000, 0.000000, 0.000000}
\pgfsetfillcolor{dialinecolor}
% was here!!!
\pgfsetarrowsend{latex}
\definecolor{dialinecolor}{rgb}{0.000000, 0.000000, 0.000000}
\pgfsetstrokecolor{dialinecolor}
\draw (20.097182\du,12.113721\du)--(20.078639\du,12.797506\du);
}
\pgfsetlinewidth{0.020000\du}
\pgfsetdash{}{0pt}
\pgfsetdash{}{0pt}
\pgfsetbuttcap
{
\definecolor{dialinecolor}{rgb}{0.000000, 0.000000, 0.000000}
\pgfsetfillcolor{dialinecolor}
% was here!!!
\pgfsetarrowsend{latex}
\definecolor{dialinecolor}{rgb}{0.000000, 0.000000, 0.000000}
\pgfsetstrokecolor{dialinecolor}
\draw (22.153535\du,12.113735\du)--(22.350263\du,12.657440\du);
}
\pgfsetlinewidth{0.020000\du}
\pgfsetdash{}{0pt}
\pgfsetdash{}{0pt}
\pgfsetbuttcap
{
\definecolor{dialinecolor}{rgb}{0.000000, 0.000000, 0.000000}
\pgfsetfillcolor{dialinecolor}
% was here!!!
\pgfsetarrowsend{latex}
\definecolor{dialinecolor}{rgb}{0.000000, 0.000000, 0.000000}
\pgfsetstrokecolor{dialinecolor}
\draw (23.647905\du,8.425855\du)--(21.317522\du,9.894865\du);
}
% setfont left to latex
\definecolor{dialinecolor}{rgb}{0.000000, 0.000000, 0.000000}
\pgfsetstrokecolor{dialinecolor}
\node[anchor=west] at (21.340000\du,8.166810\du){match 1};
\end{tikzpicture}
 &
\relsize{-2} % Graphic for TeX using PGF
% Title: /home/ameen/Desktop/match2.dia
% Creator: Dia v0.97.1
% CreationDate: Sat Jan 11 00:36:28 2014
% For: ameen
% \usepackage{tikz}
% The following commands are not supported in PSTricks at present
% We define them conditionally, so when they are implemented,
% this pgf file will use them.
\ifx\du\undefined
  \newlength{\du}
\fi
\setlength{\du}{15\unitlength}
\begin{tikzpicture}
\pgftransformxscale{1.000000}
\pgftransformyscale{-1.000000}
\definecolor{dialinecolor}{rgb}{0.000000, 0.000000, 0.000000}
\pgfsetstrokecolor{dialinecolor}
\definecolor{dialinecolor}{rgb}{1.000000, 1.000000, 1.000000}
\pgfsetfillcolor{dialinecolor}
\definecolor{dialinecolor}{rgb}{1.000000, 1.000000, 1.000000}
\pgfsetfillcolor{dialinecolor}
\fill (24.100350\du,11.350000\du)--(24.100350\du,12.250000\du)--(25.075350\du,12.250000\du)--(25.075350\du,11.350000\du)--cycle;
\pgfsetlinewidth{0.020000\du}
\pgfsetdash{}{0pt}
\pgfsetdash{}{0pt}
\pgfsetmiterjoin
\definecolor{dialinecolor}{rgb}{0.000000, 0.000000, 0.000000}
\pgfsetstrokecolor{dialinecolor}
\draw (24.100350\du,11.350000\du)--(24.100350\du,12.250000\du)--(25.075350\du,12.250000\du)--(25.075350\du,11.350000\du)--cycle;
% setfont left to latex
\definecolor{dialinecolor}{rgb}{0.000000, 0.000000, 0.000000}
\pgfsetstrokecolor{dialinecolor}
\node at (24.587850\du,11.920556\du){()};
\definecolor{dialinecolor}{rgb}{1.000000, 1.000000, 1.000000}
\pgfsetfillcolor{dialinecolor}
\pgfpathellipse{\pgfpoint{20.350327\du}{17.174574\du}}{\pgfpoint{1.216577\du}{0\du}}{\pgfpoint{0\du}{0.632074\du}}
\pgfusepath{fill}
\pgfsetlinewidth{0.020000\du}
\pgfsetdash{}{0pt}
\pgfsetdash{}{0pt}
\pgfsetmiterjoin
\definecolor{dialinecolor}{rgb}{0.000000, 0.000000, 0.000000}
\pgfsetstrokecolor{dialinecolor}
\pgfpathellipse{\pgfpoint{20.350327\du}{17.174574\du}}{\pgfpoint{1.216577\du}{0\du}}{\pgfpoint{0\du}{0.632074\du}}
\pgfusepath{stroke}
% setfont left to latex
\definecolor{dialinecolor}{rgb}{0.000000, 0.000000, 0.000000}
\pgfsetstrokecolor{dialinecolor}
\node at (20.350327\du,17.295129\du){p7=\RL{المبنى}};
\definecolor{dialinecolor}{rgb}{1.000000, 1.000000, 1.000000}
\pgfsetfillcolor{dialinecolor}
\pgfpathellipse{\pgfpoint{23.081828\du}{17.180054\du}}{\pgfpoint{1.082178\du}{0\du}}{\pgfpoint{0\du}{0.637554\du}}
\pgfusepath{fill}
\pgfsetlinewidth{0.020000\du}
\pgfsetdash{}{0pt}
\pgfsetdash{}{0pt}
\pgfsetmiterjoin
\definecolor{dialinecolor}{rgb}{0.000000, 0.000000, 0.000000}
\pgfsetstrokecolor{dialinecolor}
\pgfpathellipse{\pgfpoint{23.081828\du}{17.180054\du}}{\pgfpoint{1.082178\du}{0\du}}{\pgfpoint{0\du}{0.637554\du}}
\pgfusepath{stroke}
% setfont left to latex
\definecolor{dialinecolor}{rgb}{0.000000, 0.000000, 0.000000}
\pgfsetstrokecolor{dialinecolor}
\node at (23.081828\du,17.300610\du){r4=\RL{مقربة}};
\definecolor{dialinecolor}{rgb}{1.000000, 1.000000, 1.000000}
\pgfsetfillcolor{dialinecolor}
\fill (21.875350\du,13.465230\du)--(21.875350\du,14.380230\du)--(22.809250\du,14.380230\du)--(22.809250\du,13.465230\du)--cycle;
\pgfsetlinewidth{0.020000\du}
\pgfsetdash{}{0pt}
\pgfsetdash{}{0pt}
\pgfsetmiterjoin
\definecolor{dialinecolor}{rgb}{0.000000, 0.000000, 0.000000}
\pgfsetstrokecolor{dialinecolor}
\draw (21.875350\du,13.465230\du)--(21.875350\du,14.380230\du)--(22.809250\du,14.380230\du)--(22.809250\du,13.465230\du)--cycle;
% setfont left to latex
\definecolor{dialinecolor}{rgb}{0.000000, 0.000000, 0.000000}
\pgfsetstrokecolor{dialinecolor}
\node at (22.342300\du,14.043286\du){\^{}2};
\definecolor{dialinecolor}{rgb}{1.000000, 1.000000, 1.000000}
\pgfsetfillcolor{dialinecolor}
\pgfpathellipse{\pgfpoint{21.712850\du}{15.353540\du}}{\pgfpoint{0.537500\du}{0\du}}{\pgfpoint{0\du}{0.437500\du}}
\pgfusepath{fill}
\pgfsetlinewidth{0.020000\du}
\pgfsetdash{}{0pt}
\pgfsetdash{}{0pt}
\pgfsetmiterjoin
\definecolor{dialinecolor}{rgb}{0.000000, 0.000000, 0.000000}
\pgfsetstrokecolor{dialinecolor}
\pgfpathellipse{\pgfpoint{21.712850\du}{15.353540\du}}{\pgfpoint{0.537500\du}{0\du}}{\pgfpoint{0\du}{0.437500\du}}
\pgfusepath{stroke}
% setfont left to latex
\definecolor{dialinecolor}{rgb}{0.000000, 0.000000, 0.000000}
\pgfsetstrokecolor{dialinecolor}
\node at (21.712850\du,15.474096\du){\RL{هذا}};
\definecolor{dialinecolor}{rgb}{1.000000, 1.000000, 1.000000}
\pgfsetfillcolor{dialinecolor}
\pgfpathellipse{\pgfpoint{22.926600\du}{15.353543\du}}{\pgfpoint{0.573750\du}{0\du}}{\pgfpoint{0\du}{0.469243\du}}
\pgfusepath{fill}
\pgfsetlinewidth{0.020000\du}
\pgfsetdash{}{0pt}
\pgfsetdash{}{0pt}
\pgfsetmiterjoin
\definecolor{dialinecolor}{rgb}{0.000000, 0.000000, 0.000000}
\pgfsetstrokecolor{dialinecolor}
\pgfpathellipse{\pgfpoint{22.926600\du}{15.353543\du}}{\pgfpoint{0.573750\du}{0\du}}{\pgfpoint{0\du}{0.469243\du}}
\pgfusepath{stroke}
% setfont left to latex
\definecolor{dialinecolor}{rgb}{0.000000, 0.000000, 0.000000}
\pgfsetstrokecolor{dialinecolor}
\node at (22.926600\du,15.474098\du){\RL{من}};
\definecolor{dialinecolor}{rgb}{1.000000, 1.000000, 1.000000}
\pgfsetfillcolor{dialinecolor}
\pgfpathellipse{\pgfpoint{27.589711\du}{17.121042\du}}{\pgfpoint{1.061861\du}{0\du}}{\pgfpoint{0\du}{0.578542\du}}
\pgfusepath{fill}
\pgfsetlinewidth{0.020000\du}
\pgfsetdash{}{0pt}
\pgfsetdash{}{0pt}
\pgfsetmiterjoin
\definecolor{dialinecolor}{rgb}{0.000000, 0.000000, 0.000000}
\pgfsetstrokecolor{dialinecolor}
\pgfpathellipse{\pgfpoint{27.589711\du}{17.121042\du}}{\pgfpoint{1.061861\du}{0\du}}{\pgfpoint{0\du}{0.578542\du}}
\pgfusepath{stroke}
% setfont left to latex
\definecolor{dialinecolor}{rgb}{0.000000, 0.000000, 0.000000}
\pgfsetstrokecolor{dialinecolor}
\node at (27.589711\du,17.241597\du){p5=\RL{دبي}};
\definecolor{dialinecolor}{rgb}{1.000000, 1.000000, 1.000000}
\pgfsetfillcolor{dialinecolor}
\pgfpathellipse{\pgfpoint{25.317211\du}{17.121042\du}}{\pgfpoint{1.061861\du}{0\du}}{\pgfpoint{0\du}{0.578542\du}}
\pgfusepath{fill}
\pgfsetlinewidth{0.020000\du}
\pgfsetdash{}{0pt}
\pgfsetdash{}{0pt}
\pgfsetmiterjoin
\definecolor{dialinecolor}{rgb}{0.000000, 0.000000, 0.000000}
\pgfsetstrokecolor{dialinecolor}
\pgfpathellipse{\pgfpoint{25.317211\du}{17.121042\du}}{\pgfpoint{1.061861\du}{0\du}}{\pgfpoint{0\du}{0.578542\du}}
\pgfusepath{stroke}
% setfont left to latex
\definecolor{dialinecolor}{rgb}{0.000000, 0.000000, 0.000000}
\pgfsetstrokecolor{dialinecolor}
\node at (25.317211\du,17.241597\du){p6=\RL{مول}};
\definecolor{dialinecolor}{rgb}{1.000000, 1.000000, 1.000000}
\pgfsetfillcolor{dialinecolor}
\pgfpathellipse{\pgfpoint{24.916600\du}{15.353542\du}}{\pgfpoint{0.658750\du}{0\du}}{\pgfpoint{0\du}{0.578542\du}}
\pgfusepath{fill}
\pgfsetlinewidth{0.020000\du}
\pgfsetdash{}{0pt}
\pgfsetdash{}{0pt}
\pgfsetmiterjoin
\definecolor{dialinecolor}{rgb}{0.000000, 0.000000, 0.000000}
\pgfsetstrokecolor{dialinecolor}
\pgfpathellipse{\pgfpoint{24.916600\du}{15.353542\du}}{\pgfpoint{0.658750\du}{0\du}}{\pgfpoint{0\du}{0.578542\du}}
\pgfusepath{stroke}
% setfont left to latex
\definecolor{dialinecolor}{rgb}{0.000000, 0.000000, 0.000000}
\pgfsetstrokecolor{dialinecolor}
\node at (24.916600\du,15.474097\du){\RL{على}};
\definecolor{dialinecolor}{rgb}{1.000000, 1.000000, 1.000000}
\pgfsetfillcolor{dialinecolor}
\fill (19.827850\du,13.430230\du)--(19.827850\du,14.380230\du)--(20.752850\du,14.380230\du)--(20.752850\du,13.430230\du)--cycle;
\pgfsetlinewidth{0.020000\du}
\pgfsetdash{}{0pt}
\pgfsetdash{}{0pt}
\pgfsetmiterjoin
\definecolor{dialinecolor}{rgb}{0.000000, 0.000000, 0.000000}
\pgfsetstrokecolor{dialinecolor}
\draw (19.827850\du,13.430230\du)--(19.827850\du,14.380230\du)--(20.752850\du,14.380230\du)--(20.752850\du,13.430230\du)--cycle;
% setfont left to latex
\definecolor{dialinecolor}{rgb}{0.000000, 0.000000, 0.000000}
\pgfsetstrokecolor{dialinecolor}
\node at (20.290350\du,14.025786\du){$+$};
\definecolor{dialinecolor}{rgb}{1.000000, 1.000000, 1.000000}
\pgfsetfillcolor{dialinecolor}
\fill (19.852850\du,14.891040\du)--(19.852850\du,15.816040\du)--(20.802850\du,15.816040\du)--(20.802850\du,14.891040\du)--cycle;
\pgfsetlinewidth{0.020000\du}
\pgfsetdash{}{0pt}
\pgfsetdash{}{0pt}
\pgfsetmiterjoin
\definecolor{dialinecolor}{rgb}{0.000000, 0.000000, 0.000000}
\pgfsetstrokecolor{dialinecolor}
\draw (19.852850\du,14.891040\du)--(19.852850\du,15.816040\du)--(20.802850\du,15.816040\du)--(20.802850\du,14.891040\du)--cycle;
% setfont left to latex
\definecolor{dialinecolor}{rgb}{0.000000, 0.000000, 0.000000}
\pgfsetstrokecolor{dialinecolor}
\node at (20.327850\du,15.474096\du){$|$};
\definecolor{dialinecolor}{rgb}{1.000000, 1.000000, 1.000000}
\pgfsetfillcolor{dialinecolor}
\fill (24.475350\du,13.465230\du)--(24.475350\du,14.380230\du)--(25.486750\du,14.380230\du)--(25.486750\du,13.465230\du)--cycle;
\pgfsetlinewidth{0.020000\du}
\pgfsetdash{}{0pt}
\pgfsetdash{}{0pt}
\pgfsetmiterjoin
\definecolor{dialinecolor}{rgb}{0.000000, 0.000000, 0.000000}
\pgfsetstrokecolor{dialinecolor}
\draw (24.475350\du,13.465230\du)--(24.475350\du,14.380230\du)--(25.486750\du,14.380230\du)--(25.486750\du,13.465230\du)--cycle;
% setfont left to latex
\definecolor{dialinecolor}{rgb}{0.000000, 0.000000, 0.000000}
\pgfsetstrokecolor{dialinecolor}
\node at (24.981050\du,14.043286\du){?};
\pgfsetlinewidth{0.020000\du}
\pgfsetdash{}{0pt}
\pgfsetdash{}{0pt}
\pgfsetbuttcap
{
\definecolor{dialinecolor}{rgb}{0.000000, 0.000000, 0.000000}
\pgfsetfillcolor{dialinecolor}
% was here!!!
\pgfsetarrowsend{latex}
\definecolor{dialinecolor}{rgb}{0.000000, 0.000000, 0.000000}
\pgfsetstrokecolor{dialinecolor}
\draw (24.459070\du,12.260050\du)--(23.260613\du,16.541370\du);
}
\pgfsetlinewidth{0.020000\du}
\pgfsetdash{}{0pt}
\pgfsetdash{}{0pt}
\pgfsetbuttcap
{
\definecolor{dialinecolor}{rgb}{0.000000, 0.000000, 0.000000}
\pgfsetfillcolor{dialinecolor}
% was here!!!
\pgfsetarrowsend{latex}
\definecolor{dialinecolor}{rgb}{0.000000, 0.000000, 0.000000}
\pgfsetstrokecolor{dialinecolor}
\draw (24.090007\du,12.043880\du)--(20.763012\du,13.673686\du);
}
\pgfsetlinewidth{0.020000\du}
\pgfsetdash{}{0pt}
\pgfsetdash{}{0pt}
\pgfsetbuttcap
{
\definecolor{dialinecolor}{rgb}{0.000000, 0.000000, 0.000000}
\pgfsetfillcolor{dialinecolor}
% was here!!!
\pgfsetarrowsend{latex}
\definecolor{dialinecolor}{rgb}{0.000000, 0.000000, 0.000000}
\pgfsetstrokecolor{dialinecolor}
\draw (20.302893\du,14.389650\du)--(20.315609\du,14.880788\du);
}
\pgfsetlinewidth{0.020000\du}
\pgfsetdash{}{0pt}
\pgfsetdash{}{0pt}
\pgfsetbuttcap
{
\definecolor{dialinecolor}{rgb}{0.000000, 0.000000, 0.000000}
\pgfsetfillcolor{dialinecolor}
% was here!!!
\pgfsetarrowsend{latex}
\definecolor{dialinecolor}{rgb}{0.000000, 0.000000, 0.000000}
\pgfsetstrokecolor{dialinecolor}
\draw (20.333667\du,15.824804\du)--(20.342397\du,16.532143\du);
}
\pgfsetlinewidth{0.020000\du}
\pgfsetdash{}{0pt}
\pgfsetdash{}{0pt}
\pgfsetbuttcap
{
\definecolor{dialinecolor}{rgb}{0.000000, 0.000000, 0.000000}
\pgfsetfillcolor{dialinecolor}
% was here!!!
\pgfsetarrowsend{latex}
\definecolor{dialinecolor}{rgb}{0.000000, 0.000000, 0.000000}
\pgfsetstrokecolor{dialinecolor}
\draw (24.672711\du,12.258128\du)--(24.894654\du,13.456310\du);
}
\pgfsetlinewidth{0.020000\du}
\pgfsetdash{}{0pt}
\pgfsetdash{}{0pt}
\pgfsetbuttcap
{
\definecolor{dialinecolor}{rgb}{0.000000, 0.000000, 0.000000}
\pgfsetfillcolor{dialinecolor}
% was here!!!
\pgfsetarrowsend{latex}
\definecolor{dialinecolor}{rgb}{0.000000, 0.000000, 0.000000}
\pgfsetstrokecolor{dialinecolor}
\draw (24.959997\du,14.390119\du)--(24.943066\du,14.765987\du);
}
\pgfsetlinewidth{0.020000\du}
\pgfsetdash{}{0pt}
\pgfsetdash{}{0pt}
\pgfsetbuttcap
{
\definecolor{dialinecolor}{rgb}{0.000000, 0.000000, 0.000000}
\pgfsetfillcolor{dialinecolor}
% was here!!!
\pgfsetarrowsend{latex}
\definecolor{dialinecolor}{rgb}{0.000000, 0.000000, 0.000000}
\pgfsetstrokecolor{dialinecolor}
\draw (24.103215\du,12.258128\du)--(22.818712\du,13.472375\du);
}
\pgfsetlinewidth{0.020000\du}
\pgfsetdash{}{0pt}
\pgfsetdash{}{0pt}
\pgfsetbuttcap
{
\definecolor{dialinecolor}{rgb}{0.000000, 0.000000, 0.000000}
\pgfsetfillcolor{dialinecolor}
% was here!!!
\pgfsetarrowsend{latex}
\definecolor{dialinecolor}{rgb}{0.000000, 0.000000, 0.000000}
\pgfsetstrokecolor{dialinecolor}
\draw (22.136684\du,14.390119\du)--(21.898028\du,14.932611\du);
}
\pgfsetlinewidth{0.020000\du}
\pgfsetdash{}{0pt}
\pgfsetdash{}{0pt}
\pgfsetbuttcap
{
\definecolor{dialinecolor}{rgb}{0.000000, 0.000000, 0.000000}
\pgfsetfillcolor{dialinecolor}
% was here!!!
\pgfsetarrowsend{latex}
\definecolor{dialinecolor}{rgb}{0.000000, 0.000000, 0.000000}
\pgfsetstrokecolor{dialinecolor}
\draw (22.533168\du,14.390120\du)--(22.741011\du,14.899078\du);
}
\definecolor{dialinecolor}{rgb}{1.000000, 1.000000, 1.000000}
\pgfsetfillcolor{dialinecolor}
\fill (27.375350\du,14.891040\du)--(27.375350\du,15.816040\du)--(28.325350\du,15.816040\du)--(28.325350\du,14.891040\du)--cycle;
\pgfsetlinewidth{0.020000\du}
\pgfsetdash{}{0pt}
\pgfsetdash{}{0pt}
\pgfsetmiterjoin
\definecolor{dialinecolor}{rgb}{0.000000, 0.000000, 0.000000}
\pgfsetstrokecolor{dialinecolor}
\draw (27.375350\du,14.891040\du)--(27.375350\du,15.816040\du)--(28.325350\du,15.816040\du)--(28.325350\du,14.891040\du)--cycle;
% setfont left to latex
\definecolor{dialinecolor}{rgb}{0.000000, 0.000000, 0.000000}
\pgfsetstrokecolor{dialinecolor}
\node at (27.850350\du,15.474096\du){$|$};
\definecolor{dialinecolor}{rgb}{1.000000, 1.000000, 1.000000}
\pgfsetfillcolor{dialinecolor}
\fill (26.825350\du,13.430230\du)--(26.825350\du,14.380230\du)--(27.750350\du,14.380230\du)--(27.750350\du,13.430230\du)--cycle;
\pgfsetlinewidth{0.020000\du}
\pgfsetdash{}{0pt}
\pgfsetdash{}{0pt}
\pgfsetmiterjoin
\definecolor{dialinecolor}{rgb}{0.000000, 0.000000, 0.000000}
\pgfsetstrokecolor{dialinecolor}
\draw (26.825350\du,13.430230\du)--(26.825350\du,14.380230\du)--(27.750350\du,14.380230\du)--(27.750350\du,13.430230\du)--cycle;
% setfont left to latex
\definecolor{dialinecolor}{rgb}{0.000000, 0.000000, 0.000000}
\pgfsetstrokecolor{dialinecolor}
\node at (27.287850\du,14.025786\du){$+$};
\definecolor{dialinecolor}{rgb}{1.000000, 1.000000, 1.000000}
\pgfsetfillcolor{dialinecolor}
\fill (25.727850\du,14.891040\du)--(25.727850\du,15.816040\du)--(26.677850\du,15.816040\du)--(26.677850\du,14.891040\du)--cycle;
\pgfsetlinewidth{0.020000\du}
\pgfsetdash{}{0pt}
\pgfsetdash{}{0pt}
\pgfsetmiterjoin
\definecolor{dialinecolor}{rgb}{0.000000, 0.000000, 0.000000}
\pgfsetstrokecolor{dialinecolor}
\draw (25.727850\du,14.891040\du)--(25.727850\du,15.816040\du)--(26.677850\du,15.816040\du)--(26.677850\du,14.891040\du)--cycle;
% setfont left to latex
\definecolor{dialinecolor}{rgb}{0.000000, 0.000000, 0.000000}
\pgfsetstrokecolor{dialinecolor}
\node at (26.202850\du,15.474096\du){$|$};
\pgfsetlinewidth{0.020000\du}
\pgfsetdash{}{0pt}
\pgfsetdash{}{0pt}
\pgfsetbuttcap
{
\definecolor{dialinecolor}{rgb}{0.000000, 0.000000, 0.000000}
\pgfsetfillcolor{dialinecolor}
% was here!!!
\pgfsetarrowsend{latex}
\definecolor{dialinecolor}{rgb}{0.000000, 0.000000, 0.000000}
\pgfsetstrokecolor{dialinecolor}
\draw (27.475991\du,14.389650\du)--(27.666741\du,14.880788\du);
}
\pgfsetlinewidth{0.020000\du}
\pgfsetdash{}{0pt}
\pgfsetdash{}{0pt}
\pgfsetbuttcap
{
\definecolor{dialinecolor}{rgb}{0.000000, 0.000000, 0.000000}
\pgfsetfillcolor{dialinecolor}
% was here!!!
\pgfsetarrowsend{latex}
\definecolor{dialinecolor}{rgb}{0.000000, 0.000000, 0.000000}
\pgfsetstrokecolor{dialinecolor}
\draw (26.924947\du,14.389650\du)--(26.557011\du,14.880788\du);
}
\pgfsetlinewidth{0.020000\du}
\pgfsetdash{}{0pt}
\pgfsetdash{}{0pt}
\pgfsetbuttcap
{
\definecolor{dialinecolor}{rgb}{0.000000, 0.000000, 0.000000}
\pgfsetfillcolor{dialinecolor}
% was here!!!
\pgfsetarrowsend{latex}
\definecolor{dialinecolor}{rgb}{0.000000, 0.000000, 0.000000}
\pgfsetstrokecolor{dialinecolor}
\draw (25.084871\du,12.187535\du)--(26.815218\du,13.536712\du);
}
\pgfsetlinewidth{0.020000\du}
\pgfsetdash{}{0pt}
\pgfsetdash{}{0pt}
\pgfsetbuttcap
{
\definecolor{dialinecolor}{rgb}{0.000000, 0.000000, 0.000000}
\pgfsetfillcolor{dialinecolor}
% was here!!!
\pgfsetarrowsend{latex}
\definecolor{dialinecolor}{rgb}{0.000000, 0.000000, 0.000000}
\pgfsetstrokecolor{dialinecolor}
\draw (27.780863\du,15.824759\du)--(27.675742\du,16.537628\du);
}
\pgfsetlinewidth{0.020000\du}
\pgfsetdash{}{0pt}
\pgfsetdash{}{0pt}
\pgfsetbuttcap
{
\definecolor{dialinecolor}{rgb}{0.000000, 0.000000, 0.000000}
\pgfsetfillcolor{dialinecolor}
% was here!!!
\pgfsetarrowsend{latex}
\definecolor{dialinecolor}{rgb}{0.000000, 0.000000, 0.000000}
\pgfsetstrokecolor{dialinecolor}
\draw (25.966737\du,15.824759\du)--(25.601325\du,16.554026\du);
}
% setfont left to latex
\definecolor{dialinecolor}{rgb}{0.000000, 0.000000, 0.000000}
\pgfsetstrokecolor{dialinecolor}
\node[anchor=west] at (21.807850\du,11.829310\du){match 2};
% setfont left to latex
\definecolor{dialinecolor}{rgb}{0.000000, 0.000000, 0.000000}
\pgfsetstrokecolor{dialinecolor}
\node[anchor=west] at (19.550350\du,17.274600\du){};
\end{tikzpicture}
 \\
    \end{tabular}
} \\
\end{tabular}
}
  \caption{\framework Boolean formulae, and matches of regular expression $(P|N)+~O?~R~O\wedge 2~(P|N|U)+$}
  \label{fig:motiv}
\end{figure*}
\transtrue
\setcode{standard}

We motivate \framework with the example of Figure~\ref{fig:intromotiv} 
that shows how to extract direction entities from sample Arabic text. 
The text includes directions to a shopping mall and the 
figure presents a transliteration and an English translation of the text.
%The text contains some details that are not interesting to the direction extraction task. 
Interesting entities, such as names, places, relative directions, 
and numerical terms are highlighted with boxes in the text. 
We are interested in detecting those entities and presenting 
the direction as relational entities as 
shown in Figure~\ref{fig:intromotiv} using \framework.

%We demonstrate how to do that using \framework. 
%The user interacts with the \framework interface to specify Boolean 
%formulae, regular expressions, and semantic relations. 
%\framework will automatically extract formulae matches 
%using an in-house Arabic morphological analyzer (Sarf)
%\cite{ZaMaColing2012DemosSarf},
%extract regular expression matches, and construct the 
%semantic relations.

%For example, row 1 in the table shown in Figure~\ref{fig:motiv} 
%lists the matches of names of persons as $n_1$, $n_2$, and $n_3$. 

{\bf Entity detection.~}
As described in the table of Figure~\ref{fig:motiv}, the user denotes 
Using \framework, the user denotes the ``name of person'' entities with formula $N$ 
which requires the category feature in the morphological solution of a word
to be {\tt name of person}. 
The entities $n_1,n_2,$ and $n_3$ are matches of the formula $N$. 
Similarly, the user specifies formula $P$ to denote ``name of place'' entities. 
The user specifies formula 
$R$ to denote ``relative position'' entities, and requires the stem feature to belong 
to a selected list of 
stems containing \RL{fy} and \RL{qrb}. 
Similarly, $U$ denotes numerical terms and is a disjunction of constraints 
requiring the stem feature to belong to a set of stems such as
\RL{'wl}(first), \RL{_tAny}(second), \dots, \RL{tAs`}(ninth), and \RL{`A^sr}(tenth).

\framework calls an inhouse open source morphological 
analyzer based on finite state transducers~\cite{ZaMaColing2012DemosSarf} and computes matches of 
the formulae $N$, $P$, $R$, and $U$. 
We refer to all {\em other} words in the text that do not match 
a user defined formula as {\em null} words and we denote them by $O$. 
The resulting matches are illustrated with boxes and superscripts 
in Figure~\ref{fig:intromotiv} and are also listed in the table of 
Figure~\ref{fig:motiv}.

%{\bf Direction regular expressions.~}
The user now interacts with the regular expression editor to 
specify the direction entities and relations.
Intuitively, 
the directions are names of places ($P$) related to each other with 
positional propositions ($R$). 
A place name can be a tabulated place name, a street named after 
a person ($N$), or a numbered street ($U$). 
The text containing the directions might also include words that are not 
necessary to indicate directions ($O$)
but are necessary to complete the sentence. 

The user tries several sequences of the above entities in the 
editor and checks their matches in the visualizer. 
Finally, the user is satisfied with the matches of an expression such as 
$(P|N)+~O?~R~O\wedge 2~(P|N|U)+$ where 
$|,+,?,$ and $\wedge k$ denote disjunction, one or more, zero or one, and
up to $k$ matches, respectively.
The expression specifies a sequence of places or names of persons, 
optionally followed by a null word, 
followed by one relative position, followed by up to two possible null words, 
followed by one or more match of name of place, name of person, or numerical term. 
$O$ and $\wedge 2$ are used in the expression to allow for flexible matches. 
The user reaches the satisfying matches by experimenting with the visualizer and 
the expression editor which do not
require knowledge and expertise in regular expressions. 

\framework computes the matches of the expression in the text. 
The match trees in Figure~\ref{fig:motiv} illustrate two of them. 
The first match tree refers to the text 
\RL{brj _hlyfT bAlqrb mn AltqA.t` Al-'wl}(Khalifa Tower next to 
the first intersection). 
The second match tree refers to the text 
\RL{dby mwl `l_A mqrbT mn h_dA Almbn_A}(Dubai Mall is located near this building).
The nodes of the trees are entities and the edges and internal nodes are text, 
morphology-based, 
and word distance based relational entities. 
%The sequence and structure gives us the text of a parent node from the children nodes. 
%The same sequence can also give us the interesting (matching) morphological features. 
%The word distance is defined and abstracted by the matches of the regular expression 
%operators (internal nodes).

{\bf User defined direction relations.~} 
The user now uses the relation editor to declare user defined relations that relate 
parts of the matches of the expression with each other. 
%The aim of the semantic relations to be defined is to construct the entity-relation 
%graph shown in Figure~\ref{fig:intromotiv}. 
Intuitively, 
the user wants to create relations between places, names, and numerical entities. 
A relation between two entities can be a prepositional entity.
For example, the entities \RL{dby mwl}(Dubai Mall) and \RL{Almbn_A}(the building) are 
related by the preposition \RL{mqrbT}(near) as shown in Figure~\ref{fig:intromotiv}.

Let $e_1,o_1,r,o_2,$ and $e_2$ be the 
$(P|N)+$,$O?$,$R$,$O\wedge 2$, and $(P|N|U)+$ subexpressions, respectively.
The user selected $(P|N)+$ to be an entity after noticing in the visualizer  
that it happens to capture 
non-separated sequences of place and name entities denoting a single 
entity such as \cci{Khalifa tower}.

The declaration of \cci{Relation($e_1$,$e_2$,$r$)} creates the 
edge labeled \cci{next to} between \cci{intersection 1} and 
\cci{Khalifa tower} nodes in match 1 of Figure~\ref{fig:motiv}(b), and the 
edge labeled \cci{near} between \cci{Dubai Mall} and \cci{the building} nodes in 
match 2 of Figure~\ref{fig:motiv}(b). 
%Those relations will be constructed if $e_1$, $e_2$, and $r$ exist as match entities.
%
\cci{Relation($r$,$e_1$,$o_1$)} creates the edge labeled \cci{prep} 
between \cci{Dubai Mall} and \cci{near} nodes in match 2 of Figure~\ref{fig:motiv}(b). 
%This relation will be constructed if $r$, $e_1$, and $o_1$ exist as match entities.
%
\cci{Relation($r$,$e_2$,$o_2$)} creates the edge labeled \cci{from} 
between \cci{intersection 1} and \cci{next to} nodes in 
match 1 of Figure~\ref{fig:motiv}(b), and the 
edge labeled \cci{from this} between \cci{near} and \cci{the building} 
nodes in match 2 of Figure~\ref{fig:motiv}(b). 
%This relation will be constructed if $r$, $e_2$, and $o_2$ exist as match entities.

%\subsection*{Cross-reference} 

After constructing the user defined relations, 
the user is interested to relate the discovered entities and relational entities
that express the same concept; in particular, the same place. 
\framework provides the \cci{isA} predicate as a default cross-reference
relation which creates the edge between the nodes \cci{Khalifa Tower} and 
\cci{The building} in  Figure~\ref{fig:intromotiv}.
%The \cci{isA} predicate relates an entity $w_21$ to an entity $w_2$ two entities if they are direct synonyms 
%synonymous or one entity is a synonym of a synonym of the other entity. 
