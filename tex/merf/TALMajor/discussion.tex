The results show that \framework 
provides a friendly environment to develop entity and relational
entity extraction tasks with acceptable 
accuracy and runtime overheads compared to task specific applications. 
%
\framework requires the user to understand and interact with 
basic linguistic concepts such as readable values of morphological 
features, sequences, repetitions, and bounded repetitions. 
The user interacts with the MBF editor to specify basic concepts
and visualize their matches over highlighted text. 
%
Then, the user interacts with the MRE editor to specify 
sequences of the concepts and visualize the matches
in a graph, in conjunction with the highlighted text.

The two levels of interaction allow the user to separate between concepts 
that relate to word features, and more sophisticated entities 
that relate to sequences and context. 
%
The MBF, MRE, and user defined relations 
can be used to generate large annotated corpora in a fast manner. 
\framework visualization can be used later to edit the corpora 
and fix the annotations.
%
\major{
The case studies showed that 
\framework requires some linguistic expertise to successfully execute the 
tasks. 
In contrast, the case specific implementations require more sophisticated 
linguistic and programming expertise to attain similar results. 


We notice that ANGE, ATEEMA, and Genealogy tree report higher precision than \framework. 
This is mainly due to their capacity 
to learn words and relations that may not have a match in the 
morphological analyzer based on co-occurrence relations. 
For example, the sequence $p_1 t_1 p_2$ where $p_1$ and 
$p_2$ are persons and $t_1$ is a tell relationship helps
indicate that $x$ is a tell relationship in $p_1 x p_2$ 
even if the morphological analyzer did not return the required
feature for $x$ to match a tell relationship. 
\framework does not have that capacity yet unless it is
encoded in the C++ actions. 
}

%\subsection{ Threats to validity}
%As with any GUI tool, it takes time for the user to get acquainted with the 
%user friendly interface. 
%Relating the subexpressions to their suggested names by \framework might 
%not be directly intuitive to the user. 
%We address the above two concerns with providing the users with 
%a short tutorial video on how to use the tool.
