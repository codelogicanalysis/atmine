\section*{Reviewer 2 comments}
\textit{Comments to the Author}



The submitted paper is dealing with an interesting topic 
of knowledge extraction in context of presenting a tool 
for extracting Entities and Relationships from Arabic texts. 
The tool is mainly based on an intensive interactive process, 
where a user has to define tag types, and associates them 
with regular expressions defined over Boolean formulae, 
which are in turn defined over matches of Arabic morphological 
features extended by additional synonyms. 
A major feature of this system is providing the user with a 
user-friendly GUI, however it does not support classification 
tasks.

\noindent\answer{
  Thank you for finding our work interesting. 
}


Despite the interesting motivation and the importance of 
this topic, the presentation shows some shortcomings in the 
following sense:

\begin{enumerate}[leftmargin=0mm,label=\bfseries CommentR2.\arabic*]


\item \label{Review.2.1} 
The conceptual and implementation aspects of the framework 
were introduced in not clear and differentiated form. 
The authors are advised to separate these issues in formal 
definitions and to present the resulted software tool in 
terms of more clear software techniques.

\answer{
We introduced a Methodology Section~\ref{sec:methodology} to separate
the conceptual aspects from the implementation aspects of the
framework. 

We refer to the collection of regular expressions as a grammar, 
we also provide information about the data model used to store
the user defined tag types. 
In the results Section~\ref{sec:results} we discuss the 
complexity of the simulators. 

We now relate the action code API to corresponding software design patterns in Section~\ref{sec:relations-actions}, 
%
We discuss the translation from grammar to NDFSA. We also 
discuss the generative nature of the action code as a shared 
object library. 
}


\item \label{Review.2.2} 
Despite the motivation, that framework is proceeding from a 
morphological aspect, this feature came to short in the 
presentation; e.g. sets of morphological solutions.

\answer{
  We provide a Methodology Section~\ref{sec:methodology} 
    that benefits directly from the preceding Background.
  We introduced new Figure~\ref{f:overview} to show that 
  the morphological analysis is the first preprocessing 
  step in the data flow of \framework. 
  Section~\ref{sec:morph} that discusses morphological 
    analysis with solution examples in Table~\ref{tab:samplesolution}.
  Section XXX discussed morphological analysis in Arabic. 
  We extend that in ...
  We highlight that by referring to the Section in the 
  Introduction. 
}

\item \label{Review.2.3} 
Finally, I believe this framework should be also expressed 
in terms of the complexity and decidability problems, 
particularly when evaluating the required time of different 
development tasks. 

\answer{
  We added discussion in the Results Section~\ref{sec:results}
    to discuss and comment on the complexity of the 
    regular expression to non-deterministic finite state
    automata (NDFSA), and the simulation of the NDFSA  
    with the morphology based Boolean formulae tags. 
}

\item \label{Review.2.4} 
Providing the system with GUI is an advantage, 
However the system requires probably to some extend 
good linguistic and formal languages expertise to optimize 
the benefit of the implements tool.

\answer{
We added a statement at the end of the Discussion 
Subsection~\ref{sec:discuss} to reflect the comment of the reviewer. 
}

\end{enumerate}
