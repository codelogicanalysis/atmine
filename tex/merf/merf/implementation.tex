We implemented \framework and its GUI as a C++ desktop application 
with the Qt GUI toolkit. 
Figure~\ref{f:tagger} presents the main window of the \framework GUI.

The \framework GUI uses standard intuitive tagging facilities such as 
the application menus, the context menus, the mouse selection mechanism,
and the drag and drop mechanism to perform tagging and editing operation. 

The \framework GUI allows the user to create, manipulate, and analyze a tagging project. 
The {\em File} menu is used to create, close, or open an existing project. 
The {\em tags} menu enables the user to manually define and edit general purpose tag types.
The user can define and edit the MBF and MSF based tag types and 
trigger the automatic morphology based simulator from the {\em sarf} menu. 
The user performs comparison and analysis of two tag sets 
using the {\em Analyse} menu. 

The main window displays the text with rich colors in a pane. 
A companion pane shows a list of all the tags. 
A secondary companion pane shows details about the selected tag.

\subsection{Data Model}

The \framework tool saves the tag types and the tags in a user friendly format using
the JavaScript Object Notation (JSON) format.
The JSON interface uses a tree format based on name value pairs, where the name 
is an identifier and the value is either a sequence of values of a nested set of values. 
The nested set of values is easily defined using nested open close braces. 

The tag files keep pointers to the separate text and tag type files. 
Then they contain a list of tags with their tag type identifiers as well as their location
in text in terms of characters from the beginning of text, length of the tag, 
and in terms of word counts.

The file formats are selected to make it easier to compute statistical metrics 
necessary for CL and NLP tasks such as frequency and distance. 


\begin{figure}[h*]
\begingroup
    \fontsize{8pt}{12pt}\selectfont
\begin{verbatim}
{ "TagArray":
  [{"length":5,"pos":0,"source":1,"type":"Place"},...],
  "TagTypeFile":"/home/autoexample.stt.json",
  "file":"/home/politics.txt",
  "textchecksum":696
}
\end{verbatim}
\endgroup
\caption{Tag File Format}
\label{fig:tagfile}
\end{figure}

A sample tag file is shown in Figure~\ref{fig:tagfile}. 
The tag file includes the list of tags (TagArray), tag type file (TagTypeFile), 
text file (file), and {\em textchecksum} which is used to ensure that the 
text is not corrupt. 
A tag is defined by its position in the text (pos), length of the tag (length), 
tag type (type), and the source of the tag (source).

\begin{figure}[h*]
\begingroup
    \fontsize{8pt}{12pt}\selectfont
\begin{verbatim}
{"TagTypeSet":
  [{"Description":"Detects all the adjectives",
   "Features":[{"Stem-POS" : "ADJ" },... ],
   "Tag" : "Adjectives",
   "background_color":"#f08080",
   "bold" : false,"font" : 12,
   "foreground_color":"#778899",
   "id" : 3,"italic" : false,
   "underline" : false},...
  ]
}
\end{verbatim}
\endgroup
\caption{Tag Type File Format}
\label{fig:tagtypefile}
\end{figure}

Figure~\ref{fig:tagtypefile} shows a sample tag type file. 
The file includes the list of defined tag types (TagTypeSet). 
Each tag type is defined by the fields name (Tag), description, 
morphological features (Features), foreground color, 
background color, bold, italic, underline, font size (font), and id. 
The morphological features are stored in an array of name/value pairs.