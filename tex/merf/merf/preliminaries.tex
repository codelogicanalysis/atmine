\setcode{utf8}
\setarab

\subsection{Morphological features}

Arabic is a morphologically rich language. 
Due to the rich Arabic morphology, a single Arabic word 
might replace a complete sentence in English. 
For example, the word \RL{يَأْكُله}
stands for the English 
statement ``he/it is eating him/it''. 

Morphological analysis decomposes an Arabic word into
several morphemes,
where a morpheme is the smallest linguistic unit 
that has a meaning and fulfills a grammatical function. 
A morpheme can be a {\em stem}, or an {\em affix}. 
An affix can be a {\em prefix}, a {\em suffix}, or an {\em infix}. 
Each morpheme is associated with other morphological features 
including {\em POS}, {\em gloss}, {\em lemma}, and {\em category} tags. 
The gloss is a brief notation of the semantic meaning of a word. 
The category is a user defined tag that includes several morphemes of 
one kind. 
For example, the user can define a temporal category to include the 
prefix \RL{س} (will) and the time unit \RL{ساعة} (hour).

For example, the word \RL{يَأْكُله} is composed of three morphemes 
with associated morphological features as shown in Table~\ref{tab:samplesolution}.
The {\tt IV3MS} POS tag indicates a third person masculine singular subject pronoun 
attached to a verb, and the {\tt IVSUFF\_DO:3MS} POS tag
indicates a third person masculine
singular pronoun attached as an object to an action verb.
The {\tt VERB\_IMPERFECT} tag indicates an imperfect verb.
The notation for the gloss and POS tags is taken from the Buckwalter 
morphological analyzer~\cite{Buckwalter:02}.

%In particular, the word belongs to the category {\em Verb Imperfect}.

\begin{table}[h!]
  \centering
  \caption{Sample Solution Vector for the text \RL{يَأْكُله}.}
    \begin{tabular}{|r|c|c|c|}
          & \textbf{Prefix} & \textbf{Stem} & \textbf{Suffix} \\
    \textbf{Data} & \RL{يَ} & \RL{أْكُل} & \RL{ه} \\
    \textbf{POS} & IV3MS+ & VERB\_IMPERFECT & IVSUFF\_DO:3MS \\
    \textbf{Gloss} & he/it & eat/consume & him/it \\
    \end{tabular}%
  \label{tab:samplesolution}%
\end{table}%
\setcode{standard}

\subsection{Morphological Analyzer}
An Arabic morphological analyzer takes a word in Arabic
and returns a set of morphological solutions. 
Each solution splits the word into its morphemes and 
associates each morpheme with corresponding tags. 
Multiple solutions can be the result of multiple valid 
segmentations of the word into morphemes, or the multiple 
possible tags associated with a morpheme. 

\framework is integrated with an in-house open source Arabic morphological analyzer
based on finite state transducers~\cite{ZaMaColing2012DemosSarf}.

\subsection{Finite State Transducers}
A finite state transducer (FST) is a finite state machine with an input and an output tape. 
FSTs differ from finite state automata (FSA) in that
they have an output tape while FSAs have accept states instead.
Formally, an FST is a tuple $M = (S,S_{0},\sigma,\varGamma,\delta)$ where $S$ is the set of states,
$S_{0} \subset S$ is the set of initial states, 
$\sigma$ is the input alphabet, 
$\varGamma$ is the output alphabet, 
and $\delta \subseteq S \times (\sigma \times \{\epsilon\}) \times (\varGamma \times \{\epsilon\}) \times S$ is the transition relation. 
FSTs have been used extensively in text mining applications where the input is the text and the output is the
delimiters of a chunk of text with an associated class~\cite{beesley2001finite}. 
FSTs are attractive due to their efficiency and ease of use.

\subsection{Classes and labels}
A {\em class} is a semantic decision that the NLP of CL task tries to make. 
Parts of text that belong to the desired class are all assigned the same class {\em label}. 
For example, {\em temporal unit} is a label of a class that encompasses explicit temporal 
information in the sentence related to time such as 
\RL{dqyqT} (minute) and \RL{sA`T} (hour). 

The user may wish to define a simple class as an abstract category that contains a 
set of morphemes. 
For more sophisticated classes, the user can use the visual interface in \framework to 
define the class through a Boolean formulae with morphology based atomic terms. 
The annotation of text with the labels of the classes can later be used in 
CL and NLP tasks for learning, testing, and validation. 