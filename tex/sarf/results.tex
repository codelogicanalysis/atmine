In this section we present and discuss the results of evaluating Sarf and compare it to existing morphological 
analyzers such as SAMA and ElixirFM.%BAMA, SAMA, MADA+TOKAN, and ElixirFM.
%
%We compared the segmentation capabilities of Sarf to that of SAMA and MADA+TOKAN under the ATB v3.2. 
%We also evaluated the presence of the BAMA and SAMA inconsistencies of Tables~\ref{t:incBAMA} and \ref{t:incSAMA} in the ATB v3.2 Part 3. 
%Results show that the annotations of Sarf nearly perfectly agree with the manual annotations of the ATB v3.2. 
%The results also show that Sarf can automatically correct 0.76\% of the ATB annotations due to lexicon inconsistencies. 
%
%Moreover, we evaluated the efficiency of Sarf by measuring the size of the lexicons, lexicon augmentation cost, 
%and the runtime and accuracy performance  of the analyzer. 
%
%We compared the cost of augmenting Sarf with the question clitic (hamza \RL{'}) to that of BAMA and SAMA. 
We conducted two experiments to evaluate the performance of Sarf compared to SAMA and ElixirFM. 
The experiments show the advantage of Sarf over the other analyzers in terms of performance. 
%We also present results that compare Sarf to existing morphological analyzers such as Buckwalter and ElixirFM. 
We also evaluate our hypothesis of the utility of a customizable Arabic morphological analyzer.

%\subsection{Inconsistency}

\begin{comment}
%\vspace{-1.5cm}
\subsection{Segmentation correspondence}

We evaluated the segmentation correspondence capabilities of 
Sarf under the segmentation guidelines of the ATB v3.2.
For each entry in the `before' section of ATB v3.2 
Part 3 with a correct SAMA solution,
we automatically computed the segmentation using Sarf 
and compared the result to the segmentation in
the `after' entries that are manually validated by the LDC.

SAMA had a correct morphological 
solution for 273,618 words out of 393,201 ATB words.
Those required later segmentation and manual validation. 
Our automatically generated segmentation agree with 
99.991\% with the oracle ATB segmentation for the 273,618 words. 
%Our automatically generated segmentation agree with 99.991\% 
%of 273,618 words out of the total 393,710 ATB words 
%for which SAMA had a correct morphological solution and that 
%required later segmentation and manual validation.
We inspected the 25 entries for which our segmentation 
disagreed with ATB and found that both segmentations were valid. 
For example, the entry \utfRL{منّا} is formed of \noTrutfRL{من} 
\cci{min/PREP} (from) and \noTrutfRL{نا} \cci{nA/PRON\_1P} (us). 
Our segmentation is \noTrutfRL{من}+\noTrRL{|BBA} 
while that of ATB is \noTrutfRL{م+نّا}. 
When the morphemes are concatenated, the two \noTrRL{-n-} consonants 
can be fused into a single one with a shadda (\noTrRL{|BB}).
Since it is common to omit the shadda, we are left with a single consonant
at the boundary, which can correspond to either morpheme.

%\begin{table}[!tbp]
\begin{minipage}{\textwidth}
\begin{tabular}{lccc} 
\hline \hline
Reason & {\bf ATB}  & {\bf TOKAN} & {\bf Frequency} \\ \hline
Dropping diacritics & \utfRL{اميركياً} & \utfRL{اميركيا} &
1.456\% \\
Hamza normalization & \utfRL{أنقرة} & \utfRL{انقرة} &
2.799\% \\
Other normalizations & \utfRL{مغادرت+ه} & \utfRL{مغادرة+ه} &
5.450\% \\
Removing letters & \utfRL{لكن+ني} & \utfRL{لكن+ي} &
0.034\% \\
Adding letters & \utfRL{ل+لتحقيق} & \utfRL{ل+التحقيق} &
1.318\% \\                                                                                                              
{\bf Total} & & &{\bf 11.058\%} \\
\hline \hline
\end{tabular}
\end{minipage}
\caption{ATB-TOKAN segmentation disagreement examples}
{%\vspace{-0.5cm}
}
\label{t:tokan}
\end{table}

When we performed the same experiment using TOKAN toolkit 
(with the ATB scheme alias)
we got a total of 88.942\% agreement with ATB. 
When analyzing the inconsistent instances,
we noticed that TOKAN disregarded input diacritics. 
It also performed its segmentation based on the POS tags
of the morphological solutions in a similar approach 
to that mentioned in~\citep{LRECMaamouriKB08}. 
Table~\ref{t:tokan} shows examples of the disagreement instances.



\todo{R1.14} Since Sarf preserves correspondence when performing segmentation, it is capable of
generating a vocalized tag in the ATB `after' dataset which carries more information in
15.47\% of the time than the counterpart POS derived vocalized entry. 
The POS derived ATB vocalization processes, in some cases, the source token by splitting it into multiple unvocalized POS tokens which are then converted into vocalized ATB tokens. For example, 
the word \utfRL{للقضاء} (for/to the justice/judiciary) is split into \cci{li/PREP + Al/DET + qaDA'/NOUN} tokens. In such cases, the POS tokens differ from the source token. 
%The vocalized entry in the `after' dataset was therefore dropped. 
This issue does not exist in Sarf and the vocalized entry is maintained. 
%Since Sarf preserves correspondence when performing segmentation,
%it is capable of generating a vocalized tag in the ATB `after' dataset,
%which carries more information in 15.47\% of the time 
%than the counterpart POS derived vocalized entry.
%The vocalized entry in the `after' dataset was dropped 
%because of maintenance and segmentation issues.
%With Sarf that entry can be maintained. 

\subsection{Lexicon consistency}
\label{s:s:lexicon_consistency}
We evaluated the presence of the inconsistencies of Tables~\ref{t:incBAMA} and \ref{t:incSAMA} 
in the ATB v3.2 Part 3.
The first experiment considered the ATB entries that adopted the SAMA solution.
The rest of the entries have manually entered solutions.
The gloss inconsistencies affect 0.76\% of those entries.

\begin{table}[!tbp]
\begin{minipage}{0.6\textwidth}
\begin{tabular}{p{3cm}cc} 
\hline \hline
Conflict       & $\Delta_{word}$  & $\Delta_{analysis}$ \\ \hline
POS            & 0.206\%        & 0.016\% \\
Gloss          & 8.317\%        & 3.226\% \\
WaFa           & 0.251\%        & 0.022\% \\
%Question clitic & 0.100\% & 0.033\% \\ 
{\bf Total } & {\bf 8.774\%} & {\bf 3.264\%}\\
\hline \hline
\end{tabular}
\end{minipage}
\caption{Effect of lexicon inconsistencies.}
%\vspace{-1.5cm}
\label{t:comp:SAMA}
\end{table}
The second experiment considered all tokens in the ATB with a SAMA solution. 
The $\Delta_{word}$ column of Table~\ref{t:comp:SAMA}
reports the ratio of the affected ATB words, 
and the $\Delta_{analysis}$ columns reports the ratio 
of the conflicting morphological analyses.
One word might have several morphological 
solutions which explains the difference.
The rows report the effect of the POS and gloss tags, 
and that of the wrong prefix entry~\ref{inc:wafa} of Table~\ref{t:incSAMA}.
In total 8.774\% of the words and 3.264\% of the 
morphological solutions are affected.
Sarf automatically solves all these conflicts. 
%in addition, Sarf also automatically solves 
%the 7 ATB occurrences of the question clitic.
%\subsection{Efficiency}

\begin{table}
\begin{minipage}{\textwidth}
\begin{tabular}{lccp{0.05cm}ccp{0.05cm}cc}
\hline \hline
 & $|L_p|$ & $|R_{pp}|$ & & $|L_x|$ & $|R_{xx}|$ & & $\Delta_L^{\noTrutfRL{ء}}$ & $\Delta_R^{\noTrutfRL{ء}}$  \\
 \hline
%    \\ \cline{2-3} \cline{5-6} \cline{8-9}
{\bf BAMA}    & 299 & --& &618 & --  & & 295 & -- \\
{\bf Sarf}: Agglutinative only & 70      & 89& &181 & 123 & & 1   & 32 \\
{\bf Sarf}: Aggl. \& fusional only &                43  & 89& &146 & 128 & & 1   & 32 \\
{\bf Sarf}: Aggl., fusional, \& grouping & 41      & 7     & &146 & 32      & & 1   & 1  \\  \hline
{\bf SAMA } &    1325 & --      & &945  & -- & & 1,296 & -- \\
{\bf Sarf}: Agglutinative only & 107 & 129       & & 221 & 188& & 1     & 38     \\ 
{\bf Sarf}: Aggl. \& fusional only & 56      & 129 & & 188   & 194& & 1     & 38 \\ 
{\bf Sarf}: Aggl., fusional, \& grouping & 53      & 18 & & 188    & 64 & & 1     & 1  \\
\hline \hline
\end{tabular}
\end{minipage}
\caption{Comparison of lexicon size between BAMA, SAMA, and Sarf. Sarf is represented as BAMA or SAMA with Agglutinative, fusional, \& grouping.}
%\vspace{-2cm}
\label{t:reduction}
\end{table}

%\vspace{-2cm}
\subsection{Lexicon size.}~~
The $|L_p|$, $|L_x|$, $|R_{pp}|$, and $|R_{xx}|$ 
entries in Table~\ref{t:reduction} report 
the size of the prefix lexicon, the size of the suffix lexicon,
the number of prefix-prefix concatenation rules , 
and the number of suffix-suffix concatenation rules , respectively for BAMA and SAMA.
The entries also report the effect of using 
agglutinative affixes, fusional rules,
and grouping of rules with similar patterns using wildcards on reducing the size of the lexicon.

\todo{R1.16}
The size of the affix lexicons of Sarf is obtained by applying agglutinative affixes, fusional rules,
and grouping of rules on BAMA and SAMA affix lexicons. 
Sarf only requires 226 entries to represent the 917 entries 
of BAMA affixes with inconsistencies corrected. 
Sarf entries include 41 prefixes, seven prefix-prefix rules, 
146 suffixes, and 32 suffix-suffix rules 
while BAMA affixes include 299 prefixes and 618 suffixes.
Similarly, Sarf only requires 323 entries to represent 2,270 entries 
of SAMA affixes with inconsistencies corrected.

The transition from BAMA to SAMA required the addition of 1,353 entries to 
the lexicons of BAMA.\todo{R1.15.} 
Sarf only required the addition of one order of magnitude less entries 
to accommodate an equivalent change. 
The 136 entries consisted of 12 more entries in $L_p$, 
42 in $L_x$, 18 rules in $R_{pp}$, and 64 in $R_{xx}$.
%We also note that we detect most of the inconsistencies automatically 
%and only needed to validate our corrections in textbooks and corpora.

{\bf Augmentation.}~~
The question clitic, denoted by the glottal sign 
(hamza \utfRL{أ}), is missing in BAMA and SAMA as noted by ~\citep{Attia:06b}.
The $\Delta_L^{\noTrutfRL{ء}}$ and $\Delta_R^{\noTrutfRL{ء}}$ entries 
in Table~\ref{t:reduction} show the difference in the number of 
additional affixes and rules needed to accommodate 
for the addition of the question clitic.
Our method only requires the addition of one 
atomic affix and one fusional concatenation rule.
Whereas BAMA and SAMA need 295 and 1,296 additional entries to their lexicons, respectively, 
Moreover, the process requires manual intervention 
with a possibility of inducing inconsistencies in the process.
This is evidence that our method is better for 
the consistency and the maintenance of the lexicons.

\tododone{the table below should include the atomic features that sarf is concerned with. 
proper names, family connectors, tell-connectors, places, place prepositions, possessive.
need to recompute the values using sarf.}
\end{comment}
%\begin{table}
\begin{minipage}{\textwidth}
\relsize{-1.5} {
\begin{tabular}{lcccccc}
\hline \hline
 & \multicolumn{2}{c}{Al Kafi} & \multicolumn{2}{c}{Al Istibsar} &\multicolumn{2}{c}{Ibn Hanbal} \\ \cline{2-7}
 & Simple & Refined & Simple & Refined & Simple  & Refined \\ \hline
Word count &\multicolumn{2}{c}{98,943} & \multicolumn{2}{c}{103,835} & \multicolumn{2}{c}{20,354} \\ 
Proper names & 20,473 & 12,060 & 20,646 &  14,613 & 4,762 & 3,013\\
Family connectors & 4558 & 2544 & 7212 & 4844 & 1020 & 439 \\
Tell-connectors & 3,080 & 2,623 & 6,164 &  5,767 & 2,082 & 1,755 \\
Ignored names & 11,801 & 6,400 & 5,845 & 3,348 & 1,386 & 642 \\ \hline
Hadith accuracy & 85\% & 96\% & 96\%& 96\% & 92\%& 92\%\\ 
Chain accuracy & 99\% & 99\% & 98\% & 99\% & 99\% & 97\% \\ 
Narrator name accuracy & 93\% & 91\% &  92\% & 90\% & 90\% & 90\% \\ \hline
Name false positives & 28\% & 7\% & 9\%& 4\% & 34\% & 4\% \\ \hline
Sarf running time (secs.) & 2.97 & 1.32 & 2.93 & 1.31 & .25 & .096\\ \hline \hline
 & Buckwalter & ElixirFM & Buckwalter & ElixirFM & Buckwalter & ElixirFM \\ \hline
Proper names & 18,354 & 19,201 & 17,934 &  18,202 & 5,021 & 5,117 \\
Family connectors & 5136 & 3279 & 9445 & 6643 & 1918 & 1218 \\
Tell-connectors & 3,210 & 2,981 & 6,425 &  6,208 & 2,311 & 1,965 \\ 
Ignored names & 9,642 & 9,915 & 6,118 & 6,874 & 1,812 & 1,904 \\ \hline
Narrator name accuracy & 76\% & 86\% & 84\%& 87\% & 84\% & 84\% \\ \hline
Name false positives & 29\% & 22\% & 14\%& 13\% & 36\% & 32\% \\ \hline
Running time (secs.) & 6.65 & 2.78$\times60^2$ &4.55 & 2.3$\times60^2$ & .66 & 29.2$\times$60 \\
\hline \hline 
\end{tabular}
}
\end{minipage}
\caption{Results of the hadith case study with Sarf.}
\label{t:hadithresallresults}
\end{table}

%\begin{table}
\begin{minipage}{\textwidth}
\relsize{-1.5} {    
    \begin{tabular}{llcccccc}
\hline \hline
          &       & \multicolumn{2}{c}{Al Kafi} & \multicolumn{2}{c}{Al Istibsar} & \multicolumn{2}{c}{Ibn Hanbal} \\
\cline{3-8}
          &       & Simple & Refined & Simple & Refined & Simple & Refined \\ \hline
    \multicolumn{1}{l}{\multirow{2}[0]{*}{Proper Names}} & precision &       & 0.56  &       & 0.42  &       & 0.57 \\
    \multicolumn{1}{l}{} & recall &       & 0.94  &       & 0.95  &       & 0.98 \\ \hline
    \multicolumn{1}{l}{\multirow{2}[0]{*}{Tell connectors}} & precision &       & 0.56  &       & 0.7   &       & 0.83 \\
    \multicolumn{1}{l}{} & recall &       & 0.99  &       & 0.99  &       & 1 \\ \hline
    \multicolumn{1}{l}{\multirow{2}[0]{*}{Family connectors}} & precision &       & 0.91  &       & 0.9   &       & 1 \\
    \multicolumn{1}{l}{} & recall &       & 1     &       & 0.98  &       & 1 \\ \hline
    \multicolumn{1}{l}{\multirow{2}[0]{*}{Total}} & precision &       & 0.52  &       & 0.56  &       & 0.69 \\
    \multicolumn{1}{l}{} & recall &       & 0.97  &       & 0.71  &       & 0.99 \\ \hline
    Sarf running time (secs.) &       & 2.97  & 1.32  & 2.93  & 1.31  & 0.25  & 0.096 \\ \hline
    \multicolumn{1}{l}{\multirow{2}[0]{*}{Narrator name}} & precision &       & 0.93  &       & 0.97  &       & 0.97 \\
    \multicolumn{1}{l}{} & recall &       & 0.99  &       & 0.97  &       & 0.97 \\
    \multicolumn{1}{l}{\multirow{2}[0]{*}{Narrator chain}} & precision &       & 0.99  &       & 0.97  &       & 0.97 \\
    \multicolumn{1}{l}{} & recall &       & 0.97  &       & 1     &       & 1 \\
    \multicolumn{1}{l}{\multirow{2}[0]{*}{Hadith segmentation}} & precision &       & 0.97  &       & 0.97  &       & 0.97 \\
    \multicolumn{1}{l}{} & recall &       & 0.97  &       & 1     &       & 1 \\ \hline
          &       & SAMA  & ElixirFM & SAMA  & ElixirFM & SAMA  & ElixirFM \\
    \multicolumn{1}{l}{\multirow{2}[0]{*}{Proper Names}} & precision & 0.36  & 0.42  & 0.35  & 0.37  & 0.55  & 0.64 \\
    \multicolumn{1}{l}{} & recall & 0.83  & 0.77  & 0.81  & 0.73  & 0.79  & 0.76 \\ \hline
    \multicolumn{1}{l}{\multirow{2}[0]{*}{Tell connectors}} & precision & 0.85  & 0.84  & 0.9   & 0.92  & 0.92  & 0.94 \\
    \multicolumn{1}{l}{} & recall & 0.99  & 0.99  & 1     & 1     & 1     & 1 \\ \hline
    \multicolumn{1}{l}{\multirow{2}[0]{*}{Family connectors}} & precision & 0.65  & 0.78  & 0.73  & 0.77  & 0.9   & 0.95 \\
    \multicolumn{1}{l}{} & recall & 1     & 0.41  & 0.98  & 0.42  & 1     & 0.69 \\ \hline
    \multicolumn{1}{l}{\multirow{2}[0]{*}{Total}} & precision & 0.5   & 0.56  & 0.54  & 0.55  & 0.71  & 0.78 \\
    \multicolumn{1}{l}{} & recall & 0.92  & 0.77  & 0.91  & 0.74  & 0.9   & 0.83 \\ \hline
    Running time (secs.) &       & 6.65  & 2.78$\times60^2$ & 4.55  & 2.3$\times60^2$ & 0.66  & 29.2$\times$60 \\
    \hline \hline
\end{tabular}%
}
\end{minipage}
  \caption{Results of the hadith case study with Sarf.}
  \label{t:hadithresallresultsnew}%
\end{table}%



%\vspace{-1.2cm}
\subsection{Performance}
\label{ss:perform}

\todo{R1.5}
We evaluated the speed of Sarf and compared it to that of SAMA and ElixirFM. 
We selected a random article from an Arabic newspaper containing 1,122 words and 
computed the morphological solutions of those words using the three analyzers. 
Sarf with no application specific API implemented generates all the solutions of 
all words in 0.72 seconds while SAMA does 
the task in 2.46 seconds. 
However, ElixirFM requires much more time with 34 minutes and 19 seconds.

We also evaluated the accuracy and runtime efficiency of 
Sarf and compared that to SAMA and ElixirFM
with one of the applications~\citep{ZaMaFlairs2012HadithBio}
that used Sarf as a back-end morphological analyzer. 
The hadith extraction application~\citep{ZaMaFlairs2012HadithBio} is concerned
with analyzing a book of traditions
related to prophet Mohammad through a chain of narrators. 
%
Narrators are identified by composite proper person names
that are connected with family connectors. 
For example, the chain of narrators \RL{.hdd_tnA qtybT bn s`yd `n ^gryr}, starts with the first narrator 
\RL{qtybT bn s`yd} where \RL{qtybT} and \RL{s`yd} 
are {\tt proper person names}.
The word \RL{bn} (son of) indicates a parental 
relation that we will refer to as {\tt family-connectors}.
The name \RL{^gryr} is a proper person name denoting 
the second narrator, and the words 
\RL{.hdd_tnA} (told us) and \RL{`n} (from/about) indicate 
a narration relation and we refer to them as 
{\tt tell-connectors}. 
Key to narrator detection are morphological features that point to 
places such as names and location prepositions. 

%We calculated the execution time of the analyzers and 
%evaluated the accuracy of the tags in terms of precision and recall.
% Table generated by Excel2LaTeX from sheet 'Sheet2'
\begin{table}[tb]
\centering
%  \resizebox{0.95\textwidth}{!}{
\begin{minipage}{.8\textwidth} 
\caption{\label{tab:exp1Res}Comparison of Sarf to SAMA and ElixirFM using the hadith application}
\relsize{-2} {
  \begin{tabular}{p{1.3cm}lccccccccc}\toprule
    &       & \multicolumn{3}{c}{Al Kafi} & \multicolumn{3}{c}{Al Istibsar} & \multicolumn{3}{c}{Ibn Hanbal} \\[4pt]
          &       & Sarf  & SAMA  & ElixirFM & Sarf  & SAMA  & ElixirFM & Sarf  & SAMA  & ElixirFM \\ \colrule
    \multirow{2}{1.3cm}{Proper Names} & precision 
                                    & 0.36  & 0.36  & 0.42  & 0.38  & 0.35  & 0.37  & 0.53  & 0.55  & 0.64 \\[4pt]
    & recall 
       & 0.95  & 0.83  & 0.77  & 0.96  & 0.81  & 0.73  & 0.98  & 0.79  & 0.76 \\[6pt]
    \multirow{2}{1.3cm}{Tell connectors }
    & precision & 0.86  & 0.85  & 0.84  & 0.91   & 0.9   & 0.92  & 0.95  & 0.93  & 0.95 \\[4pt]
    & recall & 0.99  & 0.99  & 0.99  & 0.99  & 1     & 1     & 1  & 1     & 1 \\[6pt]
    \multirow{2}{1.3cm}{ Family connectors }
    & precision & 0.91  & 0.90  & 0.78  & 0.91   & 0.91  & 0.77  & 1   & 1   & 0.97 \\[4pt]
    & recall & 1     & 1     & 0.41  & 1  & 1  & 0.42  & 1  & 1  & 0.69 \\[6pt]
    \multicolumn{1}{l}{\multirow{2}[0]{*}{\bf Total}} & precision & 0.51  & 0.52  & 0.56  & 0.57  & 0.56  & 0.55  & 0.69  & 0.72  & 0.78 \\[4pt]
    \multicolumn{1}{c}{} & recall & 0.97  & 0.92  & 0.77  & 0.98  & 0.92  & 0.74  & 0.99  & 0.90   & 0.83 \\[6pt]
             {\bf Time}  & (secs)      & 1.32  & 6.65  & 2.78$\times60^2$ & 1.31  & 4.55  & 2.3$\times60^2$ & 0.096 & 0.66  & 29.2$\times$60 \\[4pt]
    \botrule
    \end{tabular}
}
\end{minipage}
%}
\end{table}%

%\begin{table}
%  \caption{Comparison of Sarf to SAMA and ElixirF using the hadith application}
%\relsize{-1.75} {
%  \begin{tabular}{l|cc|cc|cc|cc|c} 
%    \hline \hline
%    & \multicolumn{2}{c|}{Proper names} & 
%     \multicolumn{2}{c|}{Tell connectors } & 
%     \multicolumn{2}{c|}{Family connectors} & 
%     \multicolumn{2}{c|}{Total } & 
%     Time  \\ \hline
%    & Precision & Recall & Precision & Recall & Precision & Recall & Precision & Recall & (secs) \\ \hline
%  Sarf &  \\ \hline
%  SAMA &  \\ \hline
%ElixrFM&  \\ \hline
%     \hline \hline
%  \end{tabular}
%}
%%  \label{tab:exp1Res}%
%\end{table}%


Table~\ref{tab:exp1Res} reports the results of detecting morphological features 
that define proper names, tell-connectors, and family connectors and compares the 
results with SAMA and ElixirFM in terms of accuracy and running time. 
%
The table considers 
three books of hadith selected arbitrarily~\citep*{IbnHanbal,AlKulayni,AlTousi}\footnote{We obtained the digitized books from online sources such as \href{http://www.yasoob.com/}{http://www.yasoob.com/} and \href{http://www.al-eman.com/}{http://www.al-eman.com/}. }. 
All experiments used a Linux operating system running on a 
dual core 2.66 Ghz 64-bit processor with 4GB of memory.


%For each book, the table reports the accuracy of the extracted 
%proper names, tell-connectors, and family connectors in terms 
%of precision and recall for each of the compared analyzers. 
%The table also reports on the execution time 
%required by each analyzer for each book.


Sarf scored higher recall for all the features and approximately 
similar precision across the three books. 
%The precision and recall measures of the tell-connectors 
%are very close between the analyzers across the three books. 
The precision and recall measures of the family connectors 
in Sarf and SAMA are close, unlike ElixirFM which reports 
lower accuracy measures. 
\todo{R1.17}
The precision for proper names is low since some proper names exist 
in the books but outside the context of a chain of narrator names
and are hence not wanted. 
After analyzing the results, it turned out that  ElixirFM
misses some gloss tags such as the `son' tag associated with the stem `\RL{bn}'.
%This token is an important and frequent tag in our application 
%which explains the low accuracy. 
Sarf produces significantly higher proper name recall measure 
compared to SAMA and ElixirFM. This mainly due to augmenting the stem lexicon of 
Sarf with proper names as explained in Section~\ref{sec:overview}.

Sarf outperformed both SAMA and ElixirFM in running time even without the 
use of the feature priority and the feature selection API. 
SAMA performed better than ElixirFM. 
%This result was expected for Sarf as it uses structures 
%that enables the traversal in an efficient way.


%

% Table generated by Excel2LaTeX from sheet 'Sheet1'
\begin{table}
  \begin{minipage}{\textwidth}
%  \relsize{-2} {
    \begin{tabular}{l|ccc|ccc|ccc}
    \hline \hline
    \multicolumn{1}{c}{} & \multicolumn{3}{c}{Kafi} & \multicolumn{3}{c}{Istibsar} & \multicolumn{3}{c}{Ibn Hanbal} \\
    \hline
          & correct & detected & required & correct & detected & required & correct & detected & required \\
    Places & 20    & 213   & 32    & 26    & 113   & 29    & 11    & 127   & 44 \\
    Possessive & 6     & 21    & 7     & 15    & 20    & 16    & 3     & 18    & 8 \\
    vicinity of name & 8     & 180   & 13    & 18    & 99    & 22    & 9     & 111   & 20 \\
    \hline \hline
    \end{tabular}%
 %   }
    \end{minipage}
  \caption{Sarf results of extracting places, places with possessive suffix, and places in vicinity of proper names.}
  \label{tab:placesRes}%
\end{table}%



%ElixirFM outperformed Buckwalter in terms of accuracy.
%We think the big difference is due to the sophistication 
%of the formal functional morphological analysis of ElixirFM 
%and its reliance on additional annotations extracted from 
%both ATB~\citep{Maamouri:04} and the 
%Prague Arabic Dependency Treebank~\citep{Prague04}.


\subsection{Sarf API}

\begin{table}[tb]
\centering
  \begin{minipage}{.8\textwidth}
  \caption{\label{t:sarfapi}Morphological solutions per word ratio and runtime gains with the customizable Sarf API utility.}
%  \relsize{-2} {
    \begin{tabular}{lccccc}\toprule
              &             & Temporal & Hadith        & Biography       & Genealogy \\ \colrule

              & Words       & 125,010  & 18,047,732    & 14,710,064      & 21,385    \\[3pt]
   Without    & Solutions   & 4.33     & 5.41          & 5.61            & 4.63     \\
        API   & Time (secs) & 12.45    &45.21$\times$60& 133.17$\times$60& 2.89     \\[3pt]
      With    & Solutions   & 1.74     & 1.82          & 2.12            & 2.44      \\
      API     & Time (secs) & 2.39     &22.64$\times$60& 31.77$\times$60 & 0.41\\
    \botrule
    \end{tabular}%
 %   }
    \end{minipage}
\end{table}%



We evaluated the application customizable Sarf API with several applications that
used Sarf. 
\todo{R2.13}
Table~\ref{t:sarfapi} reports the number of morphological solutions per word 
and the runtime of Sarf for several applications before and after using the application customizable Sarf API.
%to refine the analysis and compares that with the same applications 
%without the use of the Sarf API. 
The numbers show the utility of the Sarf API at improving the runtime and the efficiency of 
NLP applications by an order or magnitude across the four applications. 

In what follows, we shortly describe the applications. 
Detailed results including accuracy 
measures can be found in the corresponding papers. 

The temporal entity extraction application uses finite state machines driven by
morphological features that indicate temporal units, intervals, quantities, and
prepositions as input~\citep{ZaMa2012IJCLATime}.
The application processed 43 articles arbitrarily selected from local newspapers. 
The average number of solutions that Sarf reported per word without the use of the API
was 4.33 solutions per word. 
%The use of the API indicating that 
The application specific refinements of the analysis implemented using the
Sarf API eliminated solutions that the application is not 
interested in and thus the number of solutions per word that Sarf ended up
reporting was 1.74 and that in turn resulted in a substantial improvement in runtime. 

The hadith segmentation and extraction application extracts 
chains of narrators from hadith books using finite state machines that take 
morphological features such as gloss and POS tags that indicate proper names, 
family relations, tell-connections, places, and possessive nouns as input
~\citep{ZaMaFlairs2012HadithBio}.
The application processed a total of 41 books with a total of 196,171 narrations 
to build a graph where narrators are nodes and their relation to each other are edges. 
Similarly to above, the number of solutions per word improved from 5.41 to 1.82 
and the runtime improved by more than half. 

The biography application matches a narrator extracted from the hadith application 
to corresponding biographies in biography books and extracts entities such as 
birth and death dates, location, students, professors, and authentication 
qualifications
~\citep{ZaMaFlairs2012HadithBio}.
The biography application uses narrator extraction and temporal extraction, 
and in addition
it uses morphological features that indicate qualifying adjectives that relate 
to authenticity. 
The application used the graph generated from the hadith application and 
processed 15 books of biographies with a total of 79,946 biographies 
to (1) segment the biographies, (2) extract the narrators in the biographies, and 
(3) annotate the narrator nodes in the graph with qualifiers extracted from the 
corresponding biographies. 
The use of the Sarf API improved the solutions per word ratio from 5.61 to 2.12
and improved the run time by almost a factor of 4. 

The genealogy extraction application extracts a family tree and learns words
that indicate family relations from biblical texts.
It uses morphological features that indicate proper names, family relations, places, 
and professions~\citep{ZaMaHaCicling2012Entity}. 
The application processed the book of Genesis with fifty verses. 
21,385 words. 
The use of Sarf API improved the solutions per word ratio from 4.63 to 2.44 and improved run time substantially. 



%\begin{table}
\begin{minipage}{\textwidth}
\relsize{-1} {    
    \begin{tabular}{llcccccc}
\hline \hline
          &       & \multicolumn{2}{c}{Al Kafi} & \multicolumn{2}{c}{Al Istibsar} & \multicolumn{2}{c}{Ibn Hanbal} \\
\cline{3-8}
          &       & Simple & Refined & Simple & Refined & Simple & Refined \\ \hline
    \multicolumn{1}{l}{\multirow{2}[0]{*}{Narrator name}} & precision &       & 0.93  &       & 0.97  &       & 0.97 \\
    \multicolumn{1}{l}{} & recall &       & 0.99  &       & 0.97  &       & 0.97 \\
    \multicolumn{1}{l}{\multirow{2}[0]{*}{Narrator chain}} & precision &       & 0.99  &       & 0.97  &       & 0.97 \\
    \multicolumn{1}{l}{} & recall &       & 0.97  &       & 1     &       & 1 \\
    \multicolumn{1}{l}{\multirow{2}[0]{*}{Hadith segmentation}} & precision &       & 0.97  &       & 0.97  &       & 0.97 \\
    \multicolumn{1}{l}{} & recall &       & 0.97  &       & 1     &       & 1 \\
    \hline \hline
\end{tabular}%
}
\end{minipage}
  \caption{Results of the hadith case study with Sarf.}
  \label{t:exp2Res}%
\end{table}%

%
%The concept of having an application customizable analyzer is based 
%on the hypothesis that many NLP case studies need the 
%morphological analyzer to answer queries that do not need 
%high accuracy at the low morphological analysis level.
%We conducted another experiment to validate this hypothesis.
%Thus, we extended the first application with several 
%low-level morphological refinements to Sarf such as disambiguation of 
%female proper names so that the application can ignore them. 
%We also added a refinement that detected compound proper names 
%such as \RL{`bd alr.hmn} as they are common in narrator names. 
%We inspect the effect of low level refinements on the high 
%level output and report our results in Table~\ref{tab:exp2Res}.
%
%The `narrator name' row reflects the accuracy of 
%proper names detected within narrator structures. 
%The `narrator chain' reflects the accuracy of 
%narrators covered by extracted chains. 
%The `hadith segmentation' reports on the accuracy of 
%the hadiths extracted from the book. 
%The metrics (precision and recall) are computed 
%manually over a representative sample of each book.
%
%In general, we observe that the refinements 
%slightly improved performance and accuracy at
%the lowest level of abstraction. 
%However, the refinements had little to no effect on the high 
%accuracy measures at higher levels of abstraction.
%We believe that this is a strong supporting evidence to our hypothesis.

