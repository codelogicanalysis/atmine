\documentclass[12pt,journal,compsoc]{IEEEtran}
%%%%%%%%%%%%%%%%%%%%%%
%packages 
\usepackage{tikz}
\usepackage{xspace}
\usepackage{arabtex}
\usepackage{graphicx}
%\usepackage{caption}
%\usepackage{subcaption}
\usepackage{array}
\usepackage{graphics}
\usepackage{booktabs}
\usepackage{multirow}
\usepackage{amssymb}
\usepackage{amsmath}
\usepackage{utf8}
\bibliographystyle{plain}
\usepackage{subfigure}
\usepackage{todonotes}
\usepackage{hyperref}
\usetikzlibrary{shapes,arrows}
\usepackage{color,soul}
\usepackage{verbatim}
\usepackage{array}

\usepackage{fancyvrb}
\usepackage{relsize}

\def\showvrb#1{%
\texttt{\detokenize{#1}}%
}
%%%%%%%%%%%%%%%%%%%%%%
%macros
\def\framework{\textsc{MERF}\xspace}
\newcommand{\itodo}[1]{\todo[inline,color=green]{\tiny #1}}

\newcolumntype{L}[1]{>{\raggedright\let\newline\\\arraybackslash\hspace{0pt}}m{#1}}
\newcolumntype{C}[1]{>{\centering\let\newline\\\arraybackslash\hspace{0pt}}m{#1}}
\newcolumntype{R}[1]{>{\raggedleft\let\newline\\\arraybackslash\hspace{0pt}}m{#1}}

\newcommand{\TinyRL}[1]{{\RL{#1}}}
\newcommand{\cci}[1]{{\small \texttt{#1}}}
%%%%%%%%%%%%%%%%%%%%%%%
%other 
% correct bad hyphenation here
\hyphenation{op-tical net-works semi-conduc-tor}

\begin{document}

\doublespacing

\begin{appendices}

%\tableofcontents

%\newpage

\section{Morphology}

Morphology is the study of the internal structure of an Arabic word. 
There are two types of approaches to morphology. 
The first approach is \textit{form-based morphology}. 
It is concerned with the form of the units constructing a word, 
the interactions between them, and how they form the overall structure of a word. 
The second approach is \textit{functional morphology}. 
It is concerned with the function of the units inside a word and the units' effect 
on the work syntactically and semantically.

\subsection{Form-based morphology}
The \textit{morpheme} is a central concept in form-based morphology.
Under this approach, we can distinguish \textit{concatenative} and \textit{templatic} morphemes. 

Concatenative morphology has three types of morphemes: \textit{stems}, \textit{affixes}, and \textit{clitics}. 
Stems is the base of every word, affixes attach to the beginning and the end of the stem, 
and clitics attach to the affixes. 
Affixes include \textit{prefixes}, \textit{suffixes}, and \textit{circumfixes}. 
Prefixes attach to the beginning of the stem, suffixes attach to the end of the stem, and 
circumfixes surround the stem. 
Clitics include \textit{proclitics} and \textit{enclitics}. 
The terms prefix and suffix are sometimes used to refer to proclitics and 
enclitics, respectively.

The stem can be templatic or non-templatic. 
Templatic stems are stems formed using templatic morphemes, 
while non-templatic stems can't be derived using templatic morphemes. 
Templatic morphology is concerned with the construction of the templatic stems. 
This morphology has three base morphemes that are required to construct a word templatic stem. 
The morphemes are \textit{root}, \textit{pattern}, and \textit{vocalim}. 
The root morpheme is a sequence of 3-5 consonants and 
it implies an abstract meaning common to all the derivations. 
The pattern morpheme is an abstract template used to insert both the root and vocalism. 
The vocalism morpheme specifies the short vowels that are used with a pattern.

\subsection{Functional morphology}
In functional morphology, we analyze words in terms of their 
morpho-syntactic and morpho-semantic behavior. 
This morphology has three functional operations: 
\textit{derivation}, \textit{inflection}, and \textit{cliticization}.
Derivation morphology is concerned with the creating new words from other words 
based on lexical relations such as location or time. 
Inflections provide obligatory extensions to words limited to 
a set of features such as gender and numbers. 
Cliticization is similar to inflection, 
however stands for optional features in clitics.

\section{Morphological Analysis}

Morphological analysis is the process by which we compute all the morphological solutions of a word. 
Each solution includes the choice of a single core part-of-speech for the stem or root. 
The analysis can be either form-based or functional.
\textit{Sarf} is a morphological analyzer that 
computes the morphological solutions of an input word 
based on form-based morphology and concatenative morphemes. 
It also refers to the proclitics and enclitics as prefixes and suffixes, respectively.

\end{appendices}

%\bibliography{merf}

\end{document}