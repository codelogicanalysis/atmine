\begin{table}[tb]
%\caption{Comaprison of Sarf with SAMA~\citep{Kulick:10}, 
%ElixirFM~\citep{Otakar:07}, 
%MADA+TOKAN~\citep{},
%MADAMIRA~\citep{pasha2014madamira}, 
%Beesley~\citep{}, and
%Xerox~\citep{Beesley:01}
%}
\centering
\begin{minipage}{\textwidth}
\caption{\label{tab:comparison}Comparison of Sarf with SAMA, 
ElixirFM, MADA+TOKAN, MADAMIRA, Beesley, and Fassieh}
\relsize{-1} {
  \begin{tabular}{lccccccc}\toprule
             & Sarf & SAMA & ElixirFM & MADA+TOKAN & MADAMIRA & Beesley & Fassieh\\ \colrule
Application customizable & $\checkmark$ & - & - & - &  - & - & - \\[4pt]
Feature selection & $\checkmark$ & - & - &  - & $\checkmark$ & - & - \\[4pt]
Run-on words & $\checkmark$ & - & - & - & - & - & - \\[4pt]
Partial diacritics & $\checkmark$ & - & - & - & - & $\checkmark$ & - \\[4pt]
Affix segmentation & $\checkmark$ & - & functional & tokenization schemes & statistical & - & - \\[4pt]
Root-Pattern & - & - & $\checkmark$ & - & - & $\checkmark$ & $\checkmark$ \\[4pt]
Automated disambiguation & - & - & - & SVM & SVM & - & maximum  aposteriori  \\ 
\botrule
\end{tabular}
}
\end{minipage}
\end{table}


In this section, we review work related to Arabic morphological analyzers,
segmentation correspondence, 
partial diacritics, and application specific analyzers and compare it to Sarf.

Table~\ref{tab:comparison} summarizes the comparison between Sarf 
and related Arabic morphological analyzers. 
Only ElixirFM~\citep{Otakar:07}, Beesley~\citep*{Beesley:03}, and Fassieh~\citep*{attia2009fassieh} 
provide root-pattern analysis of the stem. 
ElixirFM, MADAMIRA~\citep{pasha2014madamira}, and MADA+TOKAN~\citep*{Habash:09} are 
based on BAMA and SAMA and use functional and statistical techniques to 
address the segmentation problem by reverse engineering the multiple tags of the 
affixes. 
Sarf differs in that the segmentation is an output of the morphological
analysis and not a reverse engineering of the multi-tag affixes. 
Sarf is the only analyzer that addresses the 'run-on words' problem 
and solves it while performing the analysis. 
MADA+TOKAN, MADAMIRA, and Fassieh apply morphological disambiguation 
using support vector machines (SVM), 
and {\em maximum a posteriori} (MAP) estimation, respectively.
Beesley and Sarf consider partial diacritics to eliminate morphological 
solutions that are not in agreement with the partial diacritization. 
Sarf provides an application customizable analyzer that enables 
the developer to control and refine the analysis on the fly 
and filter the solution features. 
MADA+TOKAN and MADAMIRA provide \todo{R2.13} partial control over the output, and not the 
analysis, where MADA+TOKAN allows the user to select \todo{R2.13 } from several segmentation schemes and 
MADAMIRA enables the user to select solution features.

Sarf builds upon the lexicon of Buckwalter\citep{Buckwalter:02}.
SAMA is an updated version of BAMA with increased lexicon coverage 
and additional POS tags~\citep*{maamouri2010ldc}.
Sarf differs from Buckwalter and SAMA in that it defines 
agglutinative and fusional affixes using a shorter list of affixes
and a list of concatenation compatibility rules that allow 
prefix-prefix and suffix-suffix concatenations.
This allows Sarf to better maintain %and propagate 
the morphological tags associated with the affixes. 

Buckwalter\citep{Buckwalter:02} and SAMA~\citep{Maamouri:10} 
produce a set of segmentation solutions for
a word, compute the morphological solutions for each segment, 
compute the product of the solutions, eliminate \todo{R2.13 } the 
incompatible solutions, and then report \todo{R2.13 } the valid solutions. 
Sarf traverses the affix and stem structures with the input word
character by character and keeps a stack of morpheme nodes. 
When a morpheme node in a structure is met, 
it is checked for compatibility with the stack of nodes. 
Consequently, Sarf generates only the solutions with valid 
segmentation, and reports only those with compatible stem and affix 
concatenation. 

SAMA was refined to interact with the Arabic Treebank(ATB)~\citep{Maamouri:04}  %Arabic Treebank at the Linguistic Data Consortium 
project after the addition of a large new corpus. 
The algorithmic changes in SAMA were done manually 
to integrate with the ATB format. 
Sarf API allows for customizable refinements 
and allows Sarf to interact with any 
application on the fly
without the modification of the morphological engine itself
as was reportedly required with SAMA~\citep{Maamouri:10} 

Like ElixirFM~\citep{Otakar:07}, Sarf builds on the lexicon
of the Buckwalter analyzer. 
Sarf also uses deterministic parsing with tries and DAGs
to implement the affix and stem structures. 
We think that the inferential-realizational approach 
of ElixirFM that is highly compatible with the 
Arabic linguistic description~\citep*{Badawi:04}
can benefit from many features unique to the Arabic language.
Sarf leaves implementing that to the developer customization 
through the API since in several cases the NLP application that uses the morphological analyzer
needs only a partial linguistic model of Arabic.

MADA+TOKAN~\citep{Habash:09} is a toolkit for Arabic tokenization, diacritization, 
morphological disambiguation, POS tagging, stemming, and lemmatization. 
Sarf performs all those tasks except for morphological disambiguation 
where MADA uses SVM.
\todo{R3.6}
Sarf keeps a stack of the positions that partition text into morphemes 
as part of the solution construction process. 
Therefore, the text segments corresponding to the solution features is preserved 
with no need for further post-processing. 
%a separate segmentor such as TOKAN is not required. 

%Similar to The MADA+TOKAN toolkit, 
%Also, Sarf thinks of the several valid morphological analyses as 
%a richness that should be exploited at a higher abstract level 
%rather than an inaccuracy that should be corrected.

MADAMIRA is a tool for Arabic morphological analysis and disambiguation 
that is based on the general design of MADA, 
an Arabic morphological analyzer and disambiguator, with additional components 
inspired by AMIRA~\citep{pasha2014madamira}, a language independent SVM based analyzer. 
MADAMIRA improves upon the two systems and returns information selectively 
upon the request of the user. 
Sarf provides means for the developer to control and adapt 
the morphological analysis according to application needs. 
Moreover, the API enables the developer to implement high level 
applications such as NER which is provided by AMIRA.


%Xerox rules can be compiled into specialized finite state
%machine (FSM) based analyzers as described in~\citep{Beesley:03}.
%However, the efficiency of the resulting analyzer depends on the
%way the Xerox rules are written. 
%Writing application specific Xerox grammars and rules, or modifying the existing ones, 
%requires the NLP application developer to have deep knowledge 
%of compilation techniques, context free grammars, 
%and morphological analysis. 
%%Other morphological analyzers such as 
%%Amira~\citep{Diab:07,Benajiba:07},
%%and MAGEAD~\citep{Habash:05}%, and MADA+TOKAN~\citep{Habash:09} 
%%use machine learning and support vector machines (SVM) 
%%to enhance the accuracy of the morphological analysis at the expense 
%%of performance.

Beesley~\citep{Beesley:03} compiles 
Xerox rules into specialized finite state
machine (FSM) based morphological analyzer.
The number of machines 
generated by a compiler for Xerox rules cannot be controlled by the developer of the analyzer, 
and the composition of the FSMs into a single framework is a difficult task 
~\citep{Beesley:01}.
Consequently the efficiency of the resulting analyzer depends on the
way the Xerox rules are written. 
Writing application specific Xerox grammars and rules, or modifying the existing ones, 
requires deep knowledge and insight from the NLP application developer
in compilation techniques, context free grammars, 
and morphological analysis. 
Sarf constructs a framework of efficient structures %including 
%a trie and two DAGs 
that encode the stems and the agglutinative and fusional affixes, respectively.
\todo{R.3.15}
The structures are traversed in a manner similar to the Xerox finite state machines, 
however, they are manually optimized to reduce the number of states and the 
number of non-deterministic transitions. 
Control on non-determinism is left to the compilers with Xerox.
Sarf also provides an application customizable API that allows the developer 
to control the analysis. 
Doing the equivalent with Beesley 
requires the modification of the Xerox rules and 
the recompilation of the analyzer.
Unlike Sarf, Beesley provides a root-pattern analysis of the stem.

Fassieh is a commercial Arabic text annotation tool that 
enables the production of large Arabic text corpora~\citep{attia2009fassieh}.
The tool supports Arabic text factorization including 
morphological analysis, POS tagging, full phonetic transcription, 
and lexical semantics analysis in an automatic mode. 
Unlike Sarf, Fassieh provides morphological 
disambiguation and root-pattern analysis. 
However, Fassieh does not provide segmentation of the affix 
and reports it as a whole unit. 
This tool is not directly accessible to the research community 
and requires commercial licensing. 
Sarf differs in that it is an open-source application 
customizable tool that solves the affix segmentation and 'run-on words' problems. 

The work in~\citep*{Attia:10} addresses the detection of 
Arabic Multi-word Expressions (MWE). 
They define MWEs as 'idiosyncratic interpretations 
that cross word boundaries or spaces'. 
Sarf adopts a similar approach for specific entities such as person names 
and place names. 


Several researchers stress the importance of 
correspondence
between the input string and the tokens of the morphological 
solutions. 
Some work uses POS tags and a syntactic morphological agreement
hypothesis to refine 
syntactic boundaries within words~\citep{Regina2011}.
The work in~\citep*{LIMASemmar05}\citep{ELRASemmar08}
uses an extensive lexicon 
with 3,164,000 stems, stem rewrite rules~\citep*{Darwish:02}, 
syntax analysis, proclitics, and enclitics to address the 
same problem. 
Parallel traversal of the input string and the tokens 
of the morphological solution, while accounting for all possible
SAMA normalizations, partially solves the problem as 
reported in~\citep{LRECMaamouriKB08,lrecKulickBM10}.
Later notes in the documentation of the ATB~\citep*{ATB32LDCNum}
indicate that extensive manual work is still required and that
later versions may drop the input tokens. 
\citep{Regina2011} uses syntactic analysis to resolve the same problem. 
%A significant amount of the literature on Arabic NLP uses the Arabic Tree Bank 
%(ATB)~\citep{Maamouri:04} with tags 
%generated from BAMA and SAMA for learning and 
%evaluation~\citep{ShaalanMF10,Benajiba:07,jumaily2011}.

The survey in~\citep{Sughaiyer:04} compares
several morphological analyzers. 
Analyzers such as~\citep*{Khoja:01}\citep{Darwish:02} 
target specific applications in the analyzer itself 
or use a specific set of POS tags as their reference.
Sarf differs in that it is a general morphological 
analyzer that reports all possible solutions. 
It is application customizable in the sense that the API is used to 
control and prioritize the analysis, refine the solution features, 
and associate morphemes with developer-defined categories.

The work in \citep{Chaaben:10,Attia:00,Beesley:01} 
considers partial diacritics and performs \todo{R2.13 } morphological 
disambiguation by filtering the full morphological solutions 
and excluding inconsistent ones. 
This approach constructs several solutions that will be excluded later. 
Sarf considers diacritic consistency at the morpheme level instead of the final 
solution level. 
It checks for diacritic consistency %using Algorithm~\ref{t:consistent} 
between the input morpheme and the candidate VMF features at every 
accept node during the traversal of Sarf structures.
Sarf analysis proceeds with the consistent VMFs
and terminates the inconsistent ones.

\citep{Beesley:01}, \citep{Chaaben:10}, and \citep{Attia:00} present 
analyzers that consider partial diacritics for morphological disambiguation. 
They filter the output morphological analysis based on 
compatibility with input diacritics if found.
Sarf differs in that it considers the diacritics at morpheme boundaries 
to generate only the diacritic matching solutions, rather than 
generating all morphological solutions then filtering them.
