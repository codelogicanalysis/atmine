\begin{table}[!tbp]
\begin{minipage}{\textwidth}
\begin{tabular}{lccc} 
\hline \hline
Reason & {\bf ATB}  & {\bf TOKAN} & {\bf Frequency} \\ \hline
Dropping diacritics & \utfRL{اميركياً} & \utfRL{اميركيا} &
1.456\% \\
Hamza normalization & \utfRL{أنقرة} & \utfRL{انقرة} &
2.799\% \\
Other normalizations & \utfRL{مغادرت+ه} & \utfRL{مغادرة+ه} &
5.450\% \\
Removing letters & \utfRL{لكن+ني} & \utfRL{لكن+ي} &
0.034\% \\
Adding letters & \utfRL{ل+لتحقيق} & \utfRL{ل+التحقيق} &
1.318\% \\                                                                                                              
{\bf Total} & & &{\bf 11.058\%} \\
\hline \hline
\end{tabular}
\end{minipage}
\caption{ATB-TOKAN segmentation disagreement examples}
{%\vspace{-0.5cm}
}
\label{t:tokan}
\end{table}

When we performed the same experiment using TOKAN toolkit 
(with the ATB scheme alias)
we got a total of 88.942\% agreement with ATB. 
When analyzing the inconsistent instances,
we noticed that TOKAN disregarded input diacritics. 
It also performed its segmentation based on the POS tags
of the morphological solutions in a similar approach 
to that mentioned in~\citep{LRECMaamouriKB08}. 
Table~\ref{t:tokan} shows examples of the disagreement instances.

