%\documentstyle{llncs}
%\documentclass[12pt,fullpage]{llncs}
\documentclass[12pt]{article}
%\usepackage[left=3cm,top=1.2cm,right=3cm,nohead,nofoot]{geometry}
%\usepackage[letter,left=.25in,right=.25in,top=.17in,textwidth=7.5in,textheight=9in]{geometry}
%\usepackage[left=0cm]{geometry}

%\documentclass[12pt]{report}
%\usepackage {utthesis2}              %% Preamble.

%\usepackage{amsmath}
%\usepackage{amssymb}
\usepackage{arabtex}
\usepackage{amsmath}
\usepackage{amssymb}
\usepackage{times}
\usepackage{caption}
\usepackage{epsfig}
\usepackage{subfigure}
\usepackage{color}
\usepackage{rotate}
\usepackage{rotating}
\usepackage{multirow}
%\usepackage[colorlinks=false]{color,hyperref}
\usepackage{color,hyperref}
%\usepackage{amsthm}
\usepackage{booktabs}
\usepackage{multirow}
\usepackage{natbib}
\bibpunct{[}{]}{,}{n}{}{;}

\usepackage{url}

\usepackage{relsize}
\usepackage{fancyvrb}
\usepackage{fancyhdr,lastpage}

\usepackage{utf8}
\setarab
\fullvocalize
\transtrue
\arabtrue

\newcommand{\CharCodeIn}[1]{`\CodeIn{#1}'}
\newcommand{\CodeIn}[1]{{\small\texttt{#1}}}
\newcommand{\frl}[1]{\fbox{\RL{#1}}} 
\newcommand{\noArRL}[1]{\arabfalse\RL{#1}\arabtrue} 
\newcommand{\noTrRL}[1]{\transfalse\RL{#1}\transtrue} 
\newcommand{\noTrNoVocRL}[1]{\novocalize\transfalse\RL{#1}\transtrue\vocalize} 
%\newcommand{\drawline}{\begin{picture}(6,.1) \put(0,0) {\line(1,0){6.25}}\end{picture}}

\usepackage{setspace}
%\doublespacing
\renewcommand{\baselinestretch}{1.15}
\setlength{\parindent}{0in}
%\parskip 6pt
%\parindent 0pt
%\setlength{\parskip}{.05in}
%\oddsidemargin 0in
%\evensidemargin 0in
\oddsidemargin .0in
\evensidemargin .0in
\hoffset -.75in
\voffset -.83in
\textwidth 7.5in
\textheight 9.5in

\topsep 0in
\topmargin 0.17in

%\usepackage{arabtex}
%\usepackage{utf8}

\begin{document}


\pagestyle{fancy}
%\lhead{}
\chead{}
\rhead{NPRP No.~:~~~~4-484-1-075}

\lfoot{QNRF Form -- response to PR comments}
\cfoot{}
%\rfoot{Page \thepage~of~\pageref{LastPage} }
\rfoot{Page \thepage~of~3}
\renewcommand{\footrulewidth}{0.2pt}
\renewcommand{\headrulewidth}{0.2pt}

%\begin{titlepage}

\begin{center}
{\Large \bf Relational Arabic Text Mining Framework using 
    Morphological
    and Case-Based Analysis }

%\setlength{\unitlength}{1in}
%\setlength{\unitlength}{1in}
%\vspace{1.5in}
\vspace{.1in}

\renewcommand{\arraystretch}{.6}
\begin{tabular}{cc}
Fadi Zaraket & Rehab Duwairi \\
%Electrical and Computer Engineering &  Computer Science and Engineering Department \\
American University of Beirut & University of Qatar \\
{\tt fadi.zaraket@aub.edu.lb} & {\tt Rehab.Duwairi@qu.edu.qa}
\end{tabular}

\vspace{.1in}

\renewcommand{\arraystretch}{.6}
\begin{tabular}{c}
{\small Response to PR comments --- maximum of 3 pages} \\
\end{tabular}
\vspace{.1in}

\date{\today}
%\pagebreak
\end{center}

%\section{Background (max of 2 pages)}
%\label{s:background}
We highlight in this document the important responses to the 
excellent comments of
the respected peer reviewers (PR) of our previous Arabic text mining
framework proposal.
More detailed responses can be found in the research plan document. 
The excellent comments were instrumental in the 
progress we made and helped us bring the proposal to a far better 
quality.
We start by the changes to the proposal 
that reflect our progress after December 2009.
With that we hope to address the concern of PR2, PR3, PR4,
and PR5 on the adequacy of the research team. 

We built ATSarf~\cite{ATMine09}, 
a novel case-based morphological analyzer,
and used it to automate the analysis of 
hadith literature books. 
We extracted directed acyclic graphs (DAG) that 
represent a partial order relation between narrators and narrations.
We are currently looking at investigative and interrogative police 
reports.
We explain this work in details in Section 3 of the research document.

Our experience led us to slightly modify our approach and 
build the framework through solving several case studies.
For each case study we will visit its computational components 
and generalize them as library components. 
We will refine the components built for other case studies
and make them general enough to be used with the current case study.
We adopted a case-based approach to build a flexible
low level morphological analysis infrastructure.
The research plan document explains the approach in more details. 
This work led to manuscripts submitted for publication in 
renowned conferences. 

In the following we address the comments of the reviewers.

{\bf Concern 1. Specific Arabic features.(PR1,PR2,PR3,PR4,PR5)~~}
Reviewers liked the idea that we are going to use 
features specific to the Arabic language in our analysis and had
a concern that we did not explain how we are going to do that in 
detail. 
We mentioned previously that we are going to build relaxed
Arabic grammar rules.
We describe how to do that in Sections 4.7 and 4.9. 
We build a grammar that formalizes the Arabic language similar to 
ElixirFM~\cite{Otakar:07}.
Our grammar will be novel in that it will build relations between
morphological entities rather than
functional maps from morphological entities to solutions
using morphological trees. 
%In our approach the solutions will be expressed in a directed 
%acyclic graph
%representing a hierarchy of relations between the morphemes.
We think this provides a more succinct
representation for the solutions. Also interesting relations 
between morphemes can be computed statically before
the input is presented to the analyzer.
%Morphological trees are traversals of these DAGs and are expensive 
%to compute statically. 

We explain further in Section 4.10 page 16 how we will
use language specific features to simplify the linguistic model. 
Briefly, we will compile the formalization into finite state 
machines.
We consider the parse graph of the user input query with
elements that correspond to state variables in the FSMs of
the linguistic model. 
%The query has atomic elements corresonding to a state variables
%in the FSM.
We use this correspondence and use simple structural analysis 
techniques such as constant propagation and cone of influence
reductions to simplify the FSMs. 
This simplification is Arabic specific since
the query language matches the structure of question sentences 
in common standard Arabic.
%and the query itself reflects the way an Arabic user
%himself thinks and searches. 

{\bf Concern 2. Details about case studies (PR1,PR2,PR4,PR5).~~}
PR1 shared a common concern with other reviewers
about missing details of the case studies. 
We addressed this concern by providing extensive details to the
hadith case study in Sections 3, 4.2 and 4.10.
We also discussed the security case study with more details in 
Section 4.3.
We elaborated on the health case study in Section 4.4. 

We renamed the hadith case study study as PR1 suggested. 
We explained how authenticity will be represented as annotations
to the graph of Figure 2 of the research document. 
We leave the quantification of the annotations to the expert
scholars %to avoid the controversy of hadith interpretation
and provide a partial order of the annotations of each narrator.
The expert can decide where to draw his thresholds. 
We hope this addresses PR1's question on evaluation and quantification
of authenticity.
PR1 commented that a contradiction may not mean non authentication.
This is true and this is why the role of the tool is to flag 
inconsistencies and leave it to the expert to decide. 
PR1 also asked about additional sample queries, 
they are definitely welcome.
For example, the graph in Figure 2 allows 
for measuring the effect of one narrator 
against a certain subject covered in the hadith. 

We address the question of PR1 about how the solver works, 
in Section 4.10 as well as in the explanation of Figures 3-6 that
include the computational diagrams. 
PR1 also suggests involving experts in the project, the LPI is 
working with volunteers including experts in the hadith~\cite{Zar06},
health and security areas.

We addressed the question of
PR2 and PR5 about data gathering and data
in Sections 4.2--4.5 for each case study. 

PR4 was concerned about the relevance and semantic intricacy 
of the case studies and suggested other case studies such as
web mining, economy, and intelligence. 
%We kept our case studies as other PRs were favorable of them,
We added a web based case study in Section 4.5
and we explained how the components from the case studies 
can be used to analyze the market.
Intelligence (in the sense of machine learning) is on our future 
work list.
We understand the semantic intricacy concern of PR4 and we hope
that our preliminary results show that we can overcome 
the semantic complexities that concerned PR4. 


{\bf Concern 3. Strengthen literature review (PR1,PR2,PR3,PR4,PR5).~~}
%@inproceedings{AlSham08,
%We hope this addresses the concern of PR4 on the shortcomings
%of previous approaches and the potentials of ours. 
We went over the literature suggested by the reviewers. 
We compared our approach to current and previous approaches in the
research plan document. 
Here we refer to the references cited by PR5 and illustrate how
we made great use of them.
We hope this addresses the concern of PR4 on the shortcomings
of previous approaches and the potentials of ours. 

Daoud Daoud in~\cite{Dao09} observed that the pipeline model is not 
adequate for Arabic. 
He suggested a synchronized syntactic and morphological
approach. 
The work~\cite{AlSham08} reports a similar result where
the automated addition of syntax knowledge improved stemming
in terms of accuracy and efficiency. 
We take their arguments further to 
allow a tighter integration between the pipeline
levels up to the NLP query in question by providing a case-based
morphological analyzer. 
%However, Mr. Daoud makes a mistake when he assumes that a noun
%cannot be followed by a preposition in Arabic to resolve
%an ambiguity in his example~\noTrNoVocRL{.s-a.hb-aa s-amy _dhb-aa 
%il_A alswq} (Sami's two companions went to the market) and decide 
%that \noTrNoVocRL{_dhb-aa} is a verb (they went)
%and not a noun since it is followed by \noTrNoVocRL{il_A}.
%Surely, one can say~\noTrNoVocRL{a_h_d  .s-a.hb-aa s-amy _dhb-aa 
%il_A alswq} (Sami's two companions took gold to the market) and 
%now \noTrNoVocRL{_dhb-aa} is a noun (gold stressed with double 
%diactric).

The work of Rogati in~\cite{Rog03} builds a stemmer based on
the English stemming of 
a parallel corpus with documents containing Arabic and English
text and parallel documents containing only English with the
manual translation of Arabic. 
This approach works for parallel corpora which is not the case
for our case studies.
It is also highly dependent on the accuracy of the manual translation.
%Several interpretations of a word may be omitted if the context
%of the corpus happened to bias one meaning.

Toutanova in~\cite{Tou09} suggested that using 
a joint model to do lemmatization and POS tagging performed better
than the direct and the pipelined approach. 
We take the argument further to suggest a full joint NLP task 
model via using the NLP query to simplify and render the analysis
more accurate at the low level morphological analysis.

Later, Toutanova, in joint work with Smith and others~\cite{Smi10},
suggested better extraction quality from comparable or related
corpora. 
%They consider two sets of documents of the same nature from
%two different languages and extract parallel structures. 
This is similar to our selection of the case studies
except that our documents are in the same language.
For example, the hadith narrations are comparable to the hadith 
biographies.
%as far as the researcher is concerned with authenticity.
%The same applies to the clinical reports and the pharmaceutical 
%records, and the investigative reports from the crime scenes
%and the interrogative reports with suspects.

We benefited a lot from the work of Maamouri~\cite{Maamouri:10}
and the SAMA~\cite{Kulick:10} analyzer. 
In~\cite{Maamouri:10} Maamouri considers adding a new corpus
to the Arabic Tree Bank. The addition challenged existing
annotation guidelines. 
The solution was to refine SAMA and have it interact with the
high level NLP task.
%We find strong evidence in that to support our approach of
%a case-based morphological analyzer integrated
%with the linguistic computational model.

%@inproceedings{AlSham08,
%We hope this addresses the concern of PR4 on the shortcomings
%of previous approaches and the potentials of ours. 

%{\bf Concern 4. Continuity and charging model (PR1).~~}
%We suggested to charge on support for the framework as a partial
%means to sustain the work and not as a sole source of funds. 
%This model works fine for many open source tools as the cost 
%of open source contributions is not expensive. 
%Note that the continuation of the project does not solely depend
%on the charging model as it mostly depends on contributions to 
%research in the area from around the world.
%We will also charge on custom expertise to set up new applications.

{\bf Concern 4. Budget issues (PR1,PR2,PR4)~~}
We addressed the budget issues by reducing the requested budget.
Note that the travel and consultant expenses were noted inline 
with the itemized expense guidelines at the American
University of Beirut and at Qatar University as provided
by the corresponding research offices. 

{\bf Concern 5. List of outcomes and novelty (PR1, PR2, PR4).~~}
We refined our outcomes and detailed what is novel. 
We claim the novelty of our case-based morphological analyzer, 
a flexible linguistic computational analysis that can be loaded 
partially, the language specific optimization techniques, 
the application of language independent statistical techniques
to the simplification of the linguistic model, and the 
application of CATARM to the case studies considered. 

{\bf Concern 6. Web based case studies (PR2,PR4).~~}
We addressed the concern of PR2 and PR4 by adding a case study that 
analyzes web traffic. 
The opinion mining track suggested by PR2 is also of interest and 
can be another application we will consider.
Opinion classification and detection of spam opinions
can make use of the traffic classification modules of the network
content case study, and the spam opinion detection can 
make use of the inconsistency checkers of the hadith case study.


{\bf Concern 7. Improve evaluation (PR3).~~}
We improved the evaluation section to list how each component will 
be evaluated and what it will be compared with. 

{\bf Concern 8. Team involved (PR3).~~}
Our current team is the LPI, Co-LPI, one RA, 
and two seniors. % undergraduate students.
Once funded, the team will readily expand to 
free the students from other work and 
attract PhD and graduate students.

%{\bf Concern 10. Novelty (PR4).~~}

{\bf Concern 9. Plan for dissemination.(PR2,PR4).~~}
We devised a detailed plan to disseminate the results
to address the comments of PR4 including 
workshops and tutorial sessions to 
interest graduate students as PR2 suggested. 

{\bf Concern 10. Types of documents. (PR5)~~}
PR5 noted that analysis for documents differing in genre and 
type requires different techniques. 
We agree with that statement. 
In our case studies we target comparable documents such as 
hadith and biography and extract relations between them. 
This is similar to the work of Al-Shammari~\cite{AlSham08} that
takes as input comparable documents from two languages. 
We differ in that we consider comparable documents in the same
language.

{\bf Concern 11. Logic solvers. (PR5)~~}
PR5 asks about the role of the logic solver. 
The logic solvers will work as constraint solvers to answer 
satisfiability queries and return models from sets of solutions
when set arithmetic or logic operations are applied over
sets of morphological solutions. 
The integration of the logic solver may be deferred to later 
as PR5 suggested until the entity and relation extractors are 
available. We note that when we discuss dependency between 
work items in Section 4.12. 
However, as PR5 noted, logic solvers are close to the expertise of
the LPI and integrating them into the framework will not be
a challenging task.


{\bf Concern 12. Choice of eXtreme programming. (PR4)~~}
We explained well in our work plan in Section 4.12 why we chose 
eXtreme Programming as a methodology in response to PR4. 
For one, pair programming, encouraged by XP, 
works well in the context of academy and research to keep
expertise in the team. 
As senior students graduate, their junior partners in the pair
will keep the knowledge. 

{\bf Concern 13. Use of existing techniques. (PR5)~~}
We address this concern by citing the Buckwalter and ElixirFM 
morphological analyzer we have worked with and used so far. 
We also refer to possible use of the CADIM toolkit from Columbia
university. 
We cite all relevant techniques we plan to use including the use 
of hidden Markov models in text mining and the use of latent semantic
indexing. 

{\bf Concern 14. Relational text mining. (PR3)~~}
We explain the term {\em relational text mining} as the task of 
looking at sets of comparable documents with an expected set 
of relations between the documents. 
%Figures 1 and 2 are good examples of the relations.
A user query expresses the expected relations and can
%will be given as user queries and can 
be a complex structure of other relations and entities. 
%We gave explain that with the part

We hope we have addressed most of the concerns of the PRs in this
document and we highly appreciate their excellent and beneficial
feedback that helped us bring the proposal to a far better quality
as it stands now. 
We look forward to their positive feedback. 

\pagebreak
\rfoot{Page \thepage~of~\pageref{LastPage} }
%\bibliography{adnan_refs}
%\bibliographystyle{ieeetr}
\bibliographystyle{abbrvnat}
%\bibliographystyle{ieeetr}
{\small
\bibliography{fzAr}  
}


\end{document}
